\documentclass[12pt,a4paper]{article}

% Variables that controls behaviour
\usepackage{ifthen} % for conditional statements
\newboolean{pdflatex}
\setboolean{pdflatex}{true} % False for eps figures 

\newboolean{inbibliography}
\setboolean{inbibliography}{false} %True once you enter the bibliography


\textheight=230mm
\textwidth=160mm
\oddsidemargin=7mm
\evensidemargin=-10mm
\topmargin=-10mm
\headsep=20mm
\columnsep=5mm
\addtolength{\belowcaptionskip}{0.5em}

\renewcommand{\textfraction}{0.01}
\renewcommand{\floatpagefraction}{0.99}
\renewcommand{\topfraction}{0.9}
\renewcommand{\bottomfraction}{0.9}

\setlength{\hoffset}{-2cm}
\setlength{\voffset}{-2cm}

% Page defaults ...
\topmargin=0.5cm
\oddsidemargin=2.5cm
\textwidth=16cm
\textheight=22cm

% Don't chase after perfection
\raggedbottom
\sloppy

\usepackage{microtype}
\usepackage{lineno}    % Line numbering during drafting
\usepackage{xspace}    % Avoid problems with missing or double spaces after predefined symbold
\usepackage{caption}   % These three command get the figure and table captions automatically small
\renewcommand{\captionfont}{\small}
\renewcommand{\captionlabelfont}{\small}

%% Graphics
\usepackage{graphicx}  % to include figures (can also use other packages)
\usepackage{color}
\usepackage{colortbl}

%% Math
\usepackage{amsmath} % Adds a large collection of math symbols
\usepackage{amssymb}
\usepackage{amsfonts}
\usepackage{upgreek} % Adds in support for greek letters in roman typeset

%% fix to allow peaceful coexistence of line numbering and
%% mathematical objects
%% http://www.latex-community.org/forum/viewtopic.php?f=5&t=163
%%
\newcommand*\patchAmsMathEnvironmentForLineno[1]{%
\expandafter\let\csname old#1\expandafter\endcsname\csname #1\endcsname
\expandafter\let\csname oldend#1\expandafter\endcsname\csname
end#1\endcsname
 \renewenvironment{#1}%
   {\linenomath\csname old#1\endcsname}%
   {\csname oldend#1\endcsname\endlinenomath}%
}
\newcommand*\patchBothAmsMathEnvironmentsForLineno[1]{%
  \patchAmsMathEnvironmentForLineno{#1}%
  \patchAmsMathEnvironmentForLineno{#1*}%
}
\AtBeginDocument{%
\patchBothAmsMathEnvironmentsForLineno{equation}%
\patchBothAmsMathEnvironmentsForLineno{align}%
\patchBothAmsMathEnvironmentsForLineno{flalign}%
\patchBothAmsMathEnvironmentsForLineno{alignat}%
\patchBothAmsMathEnvironmentsForLineno{gather}%
\patchBothAmsMathEnvironmentsForLineno{multline}%
\patchBothAmsMathEnvironmentsForLineno{eqnarray}%
}

% Get hyperlinks to captions and in references.
% These do not work with revtex. Use "hypertext" as class option instead.
\usepackage{hyperref}    % Hyperlinks in references
\usepackage[all]{hypcap} % Internal hyperlinks to floats.

% Make this the last packages you include before the \begin{document}
\usepackage{cite} % Allows for ranges in citations
\usepackage{mciteplus}

\usepackage{longtable} % only for template

\begin{document}

\renewcommand{\thefootnote}{\fnsymbol{footnote}}
\setcounter{footnote}{1}

\begin{titlepage}
\pagenumbering{roman}


\vspace*{-1.5cm}
\centerline{\large THE HEP SOFTWARE FOUNDATION (HSF)}
\vspace*{1.5cm}
\noindent
\begin{tabular*}{\linewidth}{lc@{\extracolsep{\fill}}r@{\extracolsep{0pt}}}

\\
 & & HSF-TN-2015-MJF \\  % ID 
 & & \today \\ % Date - Can also hardwire e.g.: 23 March 2010
 & & \\
% not in paper \hline
\end{tabular*}

\vspace*{4.0cm}

% Title --------------------------------------------------
{\bf\boldmath\huge
\begin{center}
  Machine/Job Features
\end{center}
}

\vspace*{2.0cm}

% Authors -------------------------------------------------
\begin{center}
A.~McNab$^1$, 
Stefan Roiser$^2$, Tony Cass$^2$, Ulrich Schwickerath$^2$, ...
\bigskip\\
{\it\footnotesize
$ ^1$University of Manchester \\
$ ^2$CERN
}
\end{center}

\vspace{\fill}

% Abstract -----------------------------------------------
\begin{abstract}
  \noindent

Within the HEPiX virtualization group and the WLCG MJF Task Force, a mechanism 
has been developed which gives 
access to detailed information about the current host and the current job.
This allows user
payloads to access meta information, independent of the current batch system,
to access information like the performance of the node or calculate the
remaining run time available to them.

\end{abstract}

\vspace*{2.0cm}

\vspace{\fill}

{\footnotesize 
\centerline{\copyright~Named authors on behalf of the HSF, licence \href{http://creativecommons.org/licenses/by/4.0/}{CC-BY-4.0}.}}
\vspace*{2mm}

\end{titlepage}

% empty page may follow the title page in long documents
%\newpage
%\setcounter{page}{2}
%\mbox{~}

\cleardoublepage

\renewcommand{\thefootnote}{\arabic{footnote}}
\setcounter{footnote}{0}

%%%% Uncomment next 2 lines if desired
%\tableofcontents
%\cleardoublepage

\pagestyle{plain} % restore page numbers for the main text
\setcounter{page}{1}
\pagenumbering{arabic}

%% Uncomment during drafting and review.
%% Comment before a final submission.
\linenumbers

\section{Introduction}
\label{sec:Introduction}

The proposed schema is made to be extensible so that it can be used to add
additional information. The purpose of this document is to define the
specifications and use case of this schema. It should be seen as the source
of information for the actual implementation of the required scripts by the
sites.

\section{Definitions}
\label{sec:Definitions}

The resource provider for batch systems is the owner of the worker nodes. 
When jobs are running within virtual machines, the entity that performs
system level configuration within the VM (typically with root access) acts
as the resource provider referred to in the rest of this document.

\subsection{Environment variables}
\label{sec:EnvironmentVariables}

For each job, two environment variables may be set, with the following
names:

\begin{tabular}{l l l}
Variable	& Contents		& Comments \\
\hline
MACHINEFEATURES	& Path to a directory or URL & Information about the execution host\\
JOBFEATURES	& Path to a directory or URL & Job specific information \\
\end{tabular}

These environment variables are the base interface for the user payload.
They must be set for the job environment by the resource provider. 

In the
case of virtual machines on IaaS cloud platforms, the resource provider may
discover the values to set for the environment variables from the
machinefeatures and jobfeatures metadata keys provided by the cloud
infrastructure. These metadata keys should only be accessed once in the
lifetime of each virtual machine.

\subsection{Directories}
\label{sec:Directories}

The environment variables point to directories or URLs created by the resource
provider. Inside, the file name is the key, the contents are the values, so
that files can be referred to with expressions like
\$MACHINEFEATURES/shutdowntime . The directory name should not include the
trailing slash. These directories are either local directories in the
filesystem or sections of the URL space on an HTTP(S) server. The user
positively determines whether the files are to be opened locally or over
HTTP(S) by checking for a leading slash or the prefix http:// or https://
respectively. Typically this can achieved using library functions which can
transparently handle local files and remote URLs when opening files. The
files may be accessed multiple times to check for changes in value or in the
absence of caching by the user. The HTTP(S) server may provide HTTP cache
control and expiration information which the user may use to reduce the
number of queries.
The \$MACHINEFEATURES and \$JOBFEATURES directories contain files created by
the resource provider.
All files in the directories must be readable by both the user and the
resource provider services.

\section{Use cases}
\label{sec:Use cases}

Use cases to be covered are

\begin{tabular}{l l l p{5cm}}
Identifier & Actors & Pre-conditions & Scenario \\
\hline
1.	& user	& job starter script	& The job needs to calculate the remaining time it is allowed to run	 	  \\
2.	& user	& job starter scripts	& The job needs to know how long it was already running	 	  \\
3.	& user	& host setup		& The job wants to know the performance of the host in order to calculate the remaining time it will need to complete (for CPU intensive jobs)	 	  \\
4.	& site	& host setup		& A host needs to be drained. The payload needs to be informed of the planned shutdown time	 	  \\
5.	& site	& job starter script	& A multi-core user job on a non-exclusive node needs to know how many threads it is allowed to start. This is specifically of interest in a late-binding scenario where the pilot reserved the cores and the user payload needs to know about this.	  \\
6.	& site	& job starter script	& A user job wants to know how many job slots are allocated to the current job	  \\
7.	& site	& job starter script	& A user jobs wants to know the maximum amount of disk space it is allowed to use	  \\
8.	& site	& job starter script	& A user job wants to setup memory limits to protect itself from being killed by the batch system automatically	  \\
\end{tabular}

\section{Requirements}
\label{sec:Requirements}

\begin{itemize}
\item The proposed schema must be unique and leave no room for interpretation of the values provided.
\item For this reason, basic information is used which is well defined across sites.
\item Host and Job information can be both static (like the HS06 rating) and dynamic (eg shutdown time may be set at any time by the site).
% \item Job specific files will be owned by the user and possibly reside on a /tmp like area
\end{itemize}

\section{General explications}
\label{sec:GeneralExplications}

The implementation, that is the creation of the files and their contents, can
be highly site specific. A sample implementation can be done per batch
system in use, but it is understood that sites are allowed to change the
implementation, provided that the created numbers match the definitions
given in this document.

\section{Normalization and CPU factors}
\label{sec:Normalization}

At many sites batch resources consist of a mixture of different hardware
types which have different performance. When a user submits a job to a queue,
this queue typically sets limits on the CPU time and the wall clock time of
the job. If such a job ends up on a faster node, it will terminate quicker.
To avoid that jobs which run on slower nodes are terminated prematurely, CPU
and Wall clock times are usually scaled with a factor which depends on the
performance of the machine. This factor is called the CPU factor. A
reference machine has a CPU factor of 1. For such a machine, normalized and
real time values for CPU are the same. A CPU factor below 1 means that the
worker node is slower than a given reference. In this case normalized times
are larger than the real time values, and jobs are allowed to run longer in
order to terminate. A CPU factor above 1 means that the worker node is
faster than a given reference. In this case normalized times are smaller
than the real time values.

\section{List of requirements}
\label{sec:ListOfRequirements}

Job specific information which are:
\begin{itemize}
\item found in the directory pointed to by \$JOBFEATURES
\item owned by the user who is executing the original job. e.g. the batch
system job start script. In the case of pilots this would be the pilot user at the site.
\item created before the user job starts, eg during a job starter script.
\end{itemize}

\begin{tabular}{l l l p{2.5cm} p{5cm}}
Identifier & File Name (key) & Use cases & Value & (Optional) Comments \\
\hline
1.1	& cpufactor\_lrms		& 1,3	& Normalization factor as used by the batch system. & Can be site specific \\
1.2.1	& cpu\_limit\_secs\_lrms	& 1	& CPU limit in seconds, normalized & Divide by cpufactor\_lrms to retrieve the real time seconds. For multi-core jobs it's the total. \\
1.2.2	& cpu\_limit\_secs		& 1	& CPU limit in seconds, real time (not normalized) & For multi-core jobs it's the total. \\
1.3.1	& wall\_limit\_secs\_lrms	& 1	& Run time limit in seconds, normalized & Divide by cpufactor\_lrms to retrieve the real time seconds \\
1.3.2	& wall\_limit\_secs		& 1	& Run time limit in seconds, real time (not normalized) &  \\
1.4	& disk\_limit\_GB		& 7	& Scratch space limit in GB (if any) & If no quotas are used on a shared system, this corresponds to the full scratch space available to all jobs which run on the host. Counting is 1GB = 1000MB = 1000x1000kB \\
1.5	& jobstart\_secs		& 2	& Unix time stamp (in seconds) of the time when the job started in the batch farm. & This is what the batch system sees, not when the user payload started to work. \\
1.6	& mem\_limit\_MB		& 8	& Memory limit (if any) in MB. & Total memory. Count with 1000 not 1024, that is 4GB corresponds to 4000MB \\
1.7	& allocated\_CPU		& 5	& number of allocated cores to the current job & Allocated cores can be physical or logical \\
1.8	& shutdowntime\_job		& 1	& dynamic value, shutdown time as a UNIX time stamp (in seconds) & optional, if the file is missing no job shutdown is foreseen. The job needs to have finished all its processing when the shutdowntime has arrived \\
\end{tabular}

Host specific information which are:
\begin{itemize}
\item found in the directory pointed to by \$MACHINEFEATURES
\item readable by the user who is executing the original job. In the case of pilots this would be the pilot user at the site.
\item created before the user job starts
\end{itemize}

\begin{tabular}{l l l p{2.5cm} p{5cm}}
Identifier & File Name (key) & Use cases & Value & (Optional) Comments \\
\hline
2.1	& hs06			& 3	& HS06 rating of the full machine in it's current setup & Static value. HS06 is measured following the HEPiX recommendations. If Hyperthreading is enabled, the additional cores are treats as if they were full cores \\
2.2	& shutdowntime		& 4	& dynamic value, shutdowntime as a UNIX time stamp (in seconds) & Dynamic. If the file is missing, no shutdown is foreseen. The value is in real time, and must be in the future. Must be removed if the shutdowntime has arrived \\
2.3	& jobslots		& 6	& Number of job slots for the host & dynamic value, can change with batch reconfigurations \\
2.4	& phys\_cores		& 3	& number of physical cores &  \\
2.5	& log\_cores		& 3	& number of logical cores & can be zero if hyperthreading is off \\
2.6	& shutdown\_command	& 4	& path to a command on the machine & optional, only relevant for virtual machines. A command provided by the site which provides a hook for the user to properly destroy the virtual machine and unregister it \\
\end{tabular}

\clearpage



\section{Summary}
\label{sec:Summary}

MJF

%\section*{Acknowledgments}

%\section*{References}

%\begin{thebibliography}{9}
%% Use references in the format expected by JPCS (as used for CHEP proceedings)
%
%\bibitem{SHOULD-MUST} S. Bradner, RFC2119 ``Key words for use in RFCs to Indicate Requirement Levels'' (Internet Engineering Task Force)
%
%\bibitem{IETF-RFC} S. Bradner, RFC2026 ``The Internet Standards Process -- Revision 3'' (Internet Engineering Task Force)
%
%\end{thebibliography}


%\addcontentsline{toc}{section}{References}

\end{document}
