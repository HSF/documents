\documentclass[12pt,a4paper]{article}

% Variables that controls behaviour
\usepackage{ifthen} % for conditional statements
\newboolean{pdflatex}
\setboolean{pdflatex}{true} % False for eps figures 

\newboolean{inbibliography}
\setboolean{inbibliography}{false} %True once you enter the bibliography


\textheight=230mm
\textwidth=160mm
\oddsidemargin=7mm
\evensidemargin=-10mm
\topmargin=-10mm
\headsep=20mm
\columnsep=5mm
\addtolength{\belowcaptionskip}{0.5em}

\renewcommand{\textfraction}{0.01}
\renewcommand{\floatpagefraction}{0.99}
\renewcommand{\topfraction}{0.9}
\renewcommand{\bottomfraction}{0.9}

\setlength{\hoffset}{-2cm}
\setlength{\voffset}{-2cm}

% Page defaults ...
\topmargin=0.5cm
\oddsidemargin=2.5cm
\textwidth=16cm
\textheight=22cm

% Don't chase after perfection
\raggedbottom
\sloppy

\usepackage{float}
\usepackage{microtype}
\usepackage{lineno}    % Line numbering during drafting
\usepackage{xspace}    % Avoid problems with missing or double spaces after predefined symbold
\usepackage{caption}   % These three command get the figure and table captions automatically small
\renewcommand{\captionfont}{\small}
\renewcommand{\captionlabelfont}{\small}

\usepackage{booktabs}

%% Graphics
\usepackage{graphicx}  % to include figures (can also use other packages)
\usepackage{color}
\usepackage{colortbl}

%% Math
\usepackage{amsmath} % Adds a large collection of math symbols
\usepackage{amssymb}
\usepackage{amsfonts}
\usepackage{upgreek} % Adds in support for greek letters in roman typeset

%% fix to allow peaceful coexistence of line numbering and
%% mathematical objects
%% http://www.latex-community.org/forum/viewtopic.php?f=5&t=163
%%
\newcommand*\patchAmsMathEnvironmentForLineno[1]{%
\expandafter\let\csname old#1\expandafter\endcsname\csname #1\endcsname
\expandafter\let\csname oldend#1\expandafter\endcsname\csname
end#1\endcsname
 \renewenvironment{#1}%
   {\linenomath\csname old#1\endcsname}%
   {\csname oldend#1\endcsname\endlinenomath}%
}
\newcommand*\patchBothAmsMathEnvironmentsForLineno[1]{%
  \patchAmsMathEnvironmentForLineno{#1}%
  \patchAmsMathEnvironmentForLineno{#1*}%
}
\AtBeginDocument{%
\patchBothAmsMathEnvironmentsForLineno{equation}%
\patchBothAmsMathEnvironmentsForLineno{align}%
\patchBothAmsMathEnvironmentsForLineno{flalign}%
\patchBothAmsMathEnvironmentsForLineno{alignat}%
\patchBothAmsMathEnvironmentsForLineno{gather}%
\patchBothAmsMathEnvironmentsForLineno{multline}%
\patchBothAmsMathEnvironmentsForLineno{eqnarray}%
}

% Get hyperlinks to captions and in references.
% These do not work with revtex. Use "hypertext" as class option instead.
\usepackage{hyperref}    % Hyperlinks in references
\usepackage[all]{hypcap} % Internal hyperlinks to floats.

% Make this the last packages you include before the \begin{document}
\usepackage{cite} % Allows for ranges in citations
\usepackage{mciteplus}

\usepackage{longtable} % only for template

% Pandoc
\providecommand{\tightlist}{%
  \setlength{\itemsep}{0pt}\setlength{\parskip}{0pt}}

\begin{document}

\renewcommand{\thefootnote}{\fnsymbol{footnote}}
\setcounter{footnote}{1}

\begin{titlepage}
\pagenumbering{roman}

\vspace*{-1.5cm}
\centerline{\large THE HEP SOFTWARE FOUNDATION (HSF)}
\vspace*{1.5cm}
\noindent
\begin{tabular*}{\linewidth}{lc@{\extracolsep{\fill}}r@{\extracolsep{0pt}}}

\\
 & & HSF-TN-2020-01 v2.0 \\  % ID 
 & & 10.5281/zenodo.1469634 \\ % DOI
 & & Original Date: 4 October 2016 \\ % Date - Can also hardwire e.g.: 23 March 2010
 & & Last Update: 12 October 2023 \\ % Date - Can also hardwire e.g.: 23 March 2010
 & & \\
% not in paper \hline
\end{tabular*}

\vspace*{2.0cm}

\begin{center}
  \includegraphics*[width=4cm]{hsf_logo_angled.png}
\end{center}

\vspace*{2.0cm}


% Title --------------------------------------------------
{\bf\boldmath\huge
\begin{center}
  Proposal for HSF Project Best Practices
\end{center}
}

\vspace*{2.0cm}

% Authors -------------------------------------------------
\begin{center}
Benedikt Hegner$^1$, Benjamin Morgan$^2$, Graeme A Stewart$^1$
\bigskip\\
{\it\footnotesize
$ ^1$CERN\\
$ ^2$University of Warwick
}
\end{center}

\vspace{\fill}

\end{titlepage}

\section{Proposal for HSF Project Best
Practices}\label{proposal-for-hsf-project-best-practices}

\subsection{Introduction}\label{introduction}

This technical note is a proposed list of best practices for HSF and
other HEP open source projects. The main motivation is to ensure
interoperability and usability of a given project by other projects and
being able to build consistent software stacks. In addition, it should
make it easier for other developers to contribute to existing projects.
In the following we discuss different practices and conventions that
ease the life of

\begin{itemize}
\tightlist
\item
  developers and new contributors,
\item
  end-users and other client projects.
\end{itemize}

Afterwards we provide a checklist of the proposals you may want to use
for your project. The proposals are mainly based on experience with the
LCG software projects and releases. You may find many of the points
discussed here trivial. However, people usually differ in what they
consider trivial. The technical recommendations in this document are
tailored towards C++-based projects, but can easily be mapped onto,
e.g., Python-based projects.

\subsection{Project Scope, Name and
Visibility}\label{project-scope-name-and-visibility}

On starting a project, make sure you have an idea of the project's scope
and goals. Try to pick a name that is suitable for that. You need to
ensure uniqueness, as the name will be used to name software artifacts,
like libraries, code namespaces, error messages, etc. In addition, have
a look around to see if it conflicts with pre-existing trademarks for
software products or services.

Though it sounds like a triviality, your project should be made known to
the community. For this, having a \emph{dedicated project website} or
another entry point for information is essential. It should concentrate
all the information useful for users and developers. If possible, it
should point at all the other information listed in this document. It is
important to find the right place to put information. Try not to repeat
yourself, as duplicated documentation can easily get out of sync. Access
to all sources of project information should be granted to search engine
spiders. Furthermore, the HSF working groups and software forums allow
you to present your project or ideas at any stage in its project
lifecycle.

There is an excellent open source
\href{https://opensource.guide/}{general guide} that covers community
interactions, getting your project known and helping find new
contributors, particularly helpful for the sociology of successful open
source projects.

\subsection{Supporting Developers and
Contributors}\label{supporting-developers-and-contributors}

The following sections discuss points mainly relevant for project
developers and potential new contributors.

\subsubsection{Code repository}\label{code-repository}

The first requirement for an open-source project is fully versioned code
in a \emph{public repository}. The code should be accessible in
anonymous read-only mode by everybody. Services like
\href{https://github.com}{GitHub} or \href{https://gitlab.com}{GitLab}
provide it for free. In addition efforts like
\href{https://www.hepforge.org/}{hepforge} or labs like CERN or DESY may
host HEP-specific packages. Services supporting a clone plus
\emph{merge-request/pull-request workflow} can be extremely helpful to
attract new contributors, as it is the current de-facto standard for
open software development. Try to make sure there are no barriers to
contributing in this way, e.g., lab hosted services may be more
difficult for users without accounts to use for merge requests.

\subsubsection{License and Copyright}\label{license-and-copyright}

The ownership and copyright of the code has to be well defined and
understood. Host labs for experiments will often hold copyright on
behalf of experiments and projects that are not legal entities and this
can simplify future life enormously. In a second step, the code and
software provided should be properly licensed in order to be able to use
code provided by others, and to allow people to re-use, update, or
improve the software you provide. The
\href{http://hepsoftwarefoundation.org/technical_notes.html}{HSF
technical note} \emph{HSF-TN-2016-01} (\emph{Software Licence Agreements
HSF Policy Guidelines}) discusses various options. This is one of the
topics that is \emph{typically ignored at the beginning of a project and
hard to fix afterwards}.

\subsubsection{Compilation and other
commands}\label{compilation-and-other-commands}

Compiling, installing and testing should each be - if possible -
single-command actions. In particular, making testing easy is important.
A good place to put the necessary information is a \emph{README} file in
the repository. Relying on community standards like
\href{https://www.cmake.org}{\emph{CMake}} make it easier for others to
use and understand the setup.

\subsubsection{Testing}\label{testing}

To improve on the quality of software, unit and integration testing are
essential. Having well-documented tests makes it as well easier for
contributors to participate. They can check whether they break old
features and can, with new tests, document what their addition is
supposed to do. Testing can also assist in the triage and fixing of
bugs. Tests can be written to reproduce reported issues and fail when
they occur, with subsequent fixes validated by the tests passing.

For \emph{unit tests} plenty of software packages exist, of which
\href{https://github.com/google/googletest}{\emph{gtest}} and
\href{https://github.com/philsquared/Catch}{\emph{catch}} are two good
choices for C++ projects. Integration tests running a software project
in a certain setup can take advantage of \emph{CTest} (supplied as part
of \emph{CMake}) and \href{http://www.cdash.org}{\emph{CDash}} or be
driven by shell scripts. Ease of use is again important here, otherwise
tests tend not to be run. For example, \emph{CMake}/\emph{CTest} add
dedicated \emph{test} targets to buildscripts so that running the tests
is a simple matter of ``building'' the target, e.g.~\texttt{make\ test}
when using Makefiles.

\paragraph{Continuous integration}\label{continuous-integration}

As well as being available from the command line, tests of the code
should run on all pull/merge requests so that reviewers can immediately
see if the proposed changes break any known functionality. There are
many options to do this, well integrated with modern code repositories.

\subsubsection{Communication and
Reporting}\label{communication-and-reporting}

A mailing list to contact developers is always useful, even as issue
trackers become also more capable of supporting discussions. It is
better to have publicly and anonymously accessible archives and to be
open for subscription and posting by the public.

\subsubsection{Issue tracking}\label{issue-tracking}

It is useful to provide an issue (bug) tracker for users and developers
to interact with, allowing a view of both open and closed tickets
anonymously by the public. Optimal solutions here are the issue tracking
capabilities the code repository itself (such as GitLab or GitHub), as
solutions which integrate directly with the code repository are much
easier to use for both users and developers. CERN's
\href{https://www.atlassian.com/software/jira}{JIRA} service is an
alternative.

\subsubsection{Reference Guide}\label{reference-guide}

For developers it is important to have a good overview of provided
interfaces, existing classes, and implementation details. For this a
reference guide is a helpful tool. The de-facto standard for creating
reference guides in C++ projects is
\href{http://www.doxygen.org/}{\emph{Doxygen}}.

\subsubsection{Conventions and
Workflows}\label{conventions-and-workflows}

Every project choses certain (coding) conventions and integration
workflows. While there is a plethora of possibilities, the concretely
chosen conventions and workflows should be documented visibly. A
\emph{How to contribute} document is good practice. This is, as well, a
nice place to add information where contributions by others would be
possible and desired.

\subsubsection{Be prepared for using external (cloud)
services}\label{be-prepared-for-using-external-cloud-services}

The project should be careful in its assumptions about the environment
available for development and testing, like access to extra storage or
connections. Make it easy to integrate your software into e.g.~a
container. Containers also help greatly in deploying a continuous
integration system, e.g., testing the build on different Linux flavours.

\subsection{End-users and client
projects}\label{end-users-and-client-projects}

\subsubsection{Documentation}\label{documentation}

In addition to the already mentioned documentation, end-user focused
documentation is important. A little checklist further below summarizes
the most important information to be given as part of the documentation.

\subsubsection{Release Information}\label{release-information}

While developers (most of the time) know the changes between various
releases, it is important to document changes between releases for
end-users. It turned out to be a good policy to have multiple categories
of releases, like production releases, development releases, bug fix
releases, etc. While each project may have different conventions here,
the chosen convention should be explained, including its meaning in
terms of changes to the project's
\href{https://en.wikipedia.org/wiki/Application_programming_interface}{\emph{API}}
and
\href{https://en.wikipedia.org/wiki/Application_binary_interface}{\emph{ABI}}.
A clear numbering scheme like ``major.minor.patch'' can support this
(also known as \href{https://semver.org/}{semantic versioning}). For
each release the \emph{supported compilers}, \emph{supported operating
systems} and \emph{required dependencies} should be listed. This helps
avoiding frustrations on the user side.

\subsubsection{Interaction with
developers}\label{interaction-with-developers}

To be able to interact with developers, both the already mentioned
\emph{mailing list} and \emph{issue tracker} are important and helpful.
The required permissions to post there should be as low as possible.
Make it easy for people to give feedback and to contribute.

\subsubsection{Relocatability and co-existence of
versions}\label{relocatability-and-co-existence-of-versions}

Often a project has to be integrated into bigger software stacks. Being
relocatable, i.e.~having no hard-coded absolute paths in any build
artifact, is often a necessity to deploy and distribute these stacks. To
enable your project to become part of such a software stack, try to make
it relocatable. In addition your software should not make too strong
assumptions about its own location.

\subsubsection{Usability and run-time
settings}\label{usability-and-run-time-settings}

It should be straight forward for a user to set up and run your project.
This can for example be ensured by providing environment setup scripts,
but the number of environment variables required should be limited as
far as possible.

\subsubsection{Be prepared for using external (cloud)
services}\label{be-prepared-for-using-external-cloud-services-1}

The project should be careful in its assumptions about the user
environment, like having access to extra storage or network connections.
Like for the development process, make it easy to integrate your
software into e.g.~a container.

\subsubsection{Publications and
References}\label{publications-and-references}

Users of your software should be able to give credit to your work. Try
to publish your work in conference proceedings or journals such as
\emph{Computing and Software for Big Science} so that it can be properly
cited.

To help users cite your software make sure that you list the recommended
citation where it can easily be found in your documentation. One highly
recommended standard is now to add a \texttt{CITATION.cff} file to your
repository. This is a human and machine readable file that tells users
exactly how to write a citation. More information is on the
\href{https://citation-file-format.github.io}{Citation File Format
website}.

\subsection{Making best practices easier - the HSF Template
Project}\label{making-best-practices-easier---the-hsf-template-project}

Many of the points mentioned are per se trivial, but need some
infrastructure to be set up. To assist new projects, an HSF project
template was created. It covers many of the technical points and
provides some canonical or example implementation for many of the
issues. It is meant as open collection point of ideas and proposals by
the community.

\subsection{Checklist}\label{checklist}

A little checklist of topics to consider is given here. Not every point
applies to every project, but it may give you a handle in improving the
quality of the software you provide.

\subsubsection{Repository and code
checklist}\label{repository-and-code-checklist}

\begin{table}[H]
  \caption{Repository and code checklist}
  \label{tab:repository-and-code-checklist}
  \begin{tabular}{lll}
  \toprule
  \textbf{Topic} & \textbf{Possible solution(s)} & \textbf{Template} \\
  \midrule
Public repository & \href{https://github.com}{github},
\href{https://gitlab.com}{gitlab} & - \\
License + file & MIT, Apache2 & MIT, Apache2 \\
README file & \href{https://en.wikipedia.org/wiki/Markdown}{Markdown},
\href{http://docutils.sourceforge.net/rst.html}{reStructuredText} &
Yes \\
Reference guide & \href{http://www.doxygen.org}{Doxygen} & Doxygen \\
build scripts & \href{https://www.cmake.org}{CMake} & CMake \\
Unit testing & \href{https://github.com/google/googletest}{gtest},
\href{https://github.com/philsquared/Catch}{catch} & catch \\
Integration testing & CTest, scripts & CTest \\
version file & headers & headers \\
Relocatability & strict policy & Yes \\
environment setup & (c)sh script & - \\
\end{tabular}
\end{table}

\subsubsection{Procedure and release checklist}\label{procedure-and-release-checklist}

The following list contains mostly ``nice-to-have'' points. Having them
well-defined and documented helps both developers as well as potential
volunteer contributors.

\begin{table}[H]
  \caption{Procedure and release checklist}
  \label{tab:procedure-and-release-checklist}
  \begin{tabular}{lll}
  \toprule
  \textbf{Topic} & \textbf{Possible solution(s)} & \textbf{Template} \\
  \midrule
Defined workflow & plenty & \\
Automatic testing & \href{https://docs.github.com/en/actions}{Github Actions},
\href{https://about.gitlab.com/solutions/continuous-integration/}{gitlab CI} & - \\
Test run+reporting & CTest,CDash & CTest \\
Static Analysis &
\href{http://clang-analyzer.llvm.org}{clang-analyzer} & - \\
  \end{tabular}
\end{table}

\subsubsection{Website and information checklist}\label{website-and-information-checklist}

\begin{table}[H]
  \caption{Website and information checklist}
  \label{tab:website-and-information-checklist}
  \begin{tabular}{ll}
    \toprule
    \textbf{Topic} & \textbf{Possible solution(s)}\\
  Website & \href{https://jekyllrb.com}{jekyll},
  \href{https://pages.github.com}{github pages}, \href{https://gohugo.io}{Hugo} \\
  How to contribute & - \\
  User manual & markdown, doxygen \\
  Reference manual & doxygen \\
  Bug tracker & github, gitlab, jira \\
  Mailing list & google groups, e-groups \\
  Link to repository & - \\
  List of releases & - \\
  List of supported OS+compilers & - \\
  List of pre-requisites & - \\
  CITATION.cff file & \href{https://citation-file-format.github.io}{CFF website} \\
  \end{tabular}
\end{table}

\subsection{Summary and Outlook}\label{summary-and-outlook}

This document described a few best practices, and potential
implementations. Updates, additional points or corrections are very
welcome.

\end{document}
