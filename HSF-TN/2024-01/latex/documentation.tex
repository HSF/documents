\section{End user documentation}
\label{documentation}

The ability for all users, regardless of technical capability, to use and profit from AFs is crucial. The necessary developments in analysis techniques and infrastructure must therefore be undertaken in collaboration with analysts to keep the “barrier-to-entry” on using new analysis facilities/infrastructures as low as possible. Analysts will expect comprehensive documentation and, more importantly, tutorials on how to run analyses that are similar to theirs on the facility - as an example the IRIS-HEP Analysis Grand Challenge has complete analysis workflows, including the configuration of facilities, using open-source code and CERN Open Data that can serve as valuable instructional resources. Tutorials for analysis specific tasks that illustrate how to most effectively use AFs will also be required. These tutorials should be regularly tested on the facilities (perhaps as part of a periodic testing schedule). We can target experiments’ training and onboarding initiatives such as the LHCb Starterkit and the CMS Data Analysis School, and enlist the help of the HSF training working group~\cite{hsftraining}. These initiatives/groups can provide dedicated lessons with hands-on tutorials to train early career researchers on these infrastructures from the outset --- these initiatives are documented in~\cite{Hageboeck:2023kcb}. The findability, and usability of this documentation and instructional material should be considered an important user metric. It is clear that such educational material needs to rely on a degree of uniformity between the user interfaces --- hiding, as much as possible, the differences between facilities --- to avoid filling the documentation with caveats.

End-user support can be provided both by the facility and by peers. Forums such as Slack, Mattermost and Discord channels provide vital, realtime, peer-to-peer support among countless communities and could be utilized for end-user support on a facility. The facility should endorse such a channel's existence and direct users to it as an official source of support.
