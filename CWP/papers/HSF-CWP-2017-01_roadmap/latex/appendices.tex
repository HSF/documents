\begin{appendices}

\hypertarget{appendix-a---list-of-workshops}{%
\section{List of Workshops}\label{appendix-a---list-of-workshops}}

\textbf{HEP Software Foundation Workshop}\\
\emph{Date:} 23-26 Jan, 2017\\
\emph{Location:} UCSD/SDSC (La Jolla, CA, USA)\\
\emph{URL:}
\href{http://indico.cern.ch/event/570249/}{{http://indico.cern.ch/event/570249/}}\\
\emph{Description:} This HSF workshop at SDSC/UCSD was the first
workshop supporting the CWP process. There were plenary sessions
covering topics of general interest as well as parallel sessions for the
many topical working groups in progress for the CWP.\\

\noindent
\textbf{Software Triggers and Event Reconstruction WG meeting}\\
\emph{Date:} 9 Mar, 2017\\
\emph{Location:} LAL-Orsay (Orsay, France)\\
\emph{URL:}
\href{https://indico.cern.ch/event/614111/}{{https://indico.cern.ch/event/614111/}}\\
\emph{Description:} This was a meeting of the Software Triggers and Event
Reconstruction CWP working group. It was held as a parallel session at
the ``Connecting the Dots'' workshop, which focuses on forward-looking
pattern recognition and machine learning algorithms for use in HEP.\\

\noindent
\textbf{IML Topical Machine Learning Workshop}\\
\emph{Date:} 20-22 Mar, 2017\\
\emph{Location:} CERN (Geneva, Switzerland)\\
\emph{URL:}
\href{https://indico.cern.ch/event/595059}{{https://indico.cern.ch/event/595059}}\\
\emph{Description:} This was a meeting of the Machine Learning CWP
working group. It was held as a parallel session at the
``Inter-experimental Machine Learning (IML)'' workshop, an organisation
formed in 2016 to facilitate communication regarding R\&D on ML
applications in the LHC experiments.\\

\noindent
\textbf{Community White Paper Follow-up at FNAL}\\
\emph{Date:} 23 Mar, 2017\\
\emph{Location:} FNAL (Batavia, IL, USA)\\
\emph{URL:}
\href{https://indico.fnal.gov/conferenceDisplay.py?confId=14032}{{https://indico.fnal.gov/conferenceDisplay.py?confId=14032}}\\
\emph{Description:} This one-day workshop was organised to engage with
the experimental HEP community involved in computing and software for
Intensity Frontier experiments at FNAL. Plans for the CWP were
described, with discussion about commonalities between the HL-LHC
challenges and the challenges of the FNAL neutrino and muon
experiments\\

\noindent
\textbf{CWP Visualisation Workshop}\\
\emph{Date:} 28-30 Mar, 2017\\
\emph{Location:} CERN (Geneva, Switzerland)\\
\emph{URL:}
\href{https://indico.cern.ch/event/617054/}{{https://indico.cern.ch/event/617054/}}\\
\emph{Description:} This workshop was organised by the Visualisation CWP
working group. It explored the current landscape of HEP visualisation
tools as well as visions for how these could evolve. There was
participation both from HEP developers and industry.\\

\noindent
\textbf{DS@HEP 2017 (Data Science in High Energy Physics)}\\
\emph{Date:} 8-12 May, 2017\\
\emph{Location:} FNAL (Batava, IL, USA)\\
\emph{URL:}
\href{https://indico.fnal.gov/conferenceDisplay.py?confId=13497}{{https://indico.fnal.gov/conferenceDisplay.py?confId=13497}}\\
\emph{Description:} This was a meeting of the Machine Learning CWP
working group. It was held as a parallel session at the ``Data Science
in High Energy Physics (DS@HEP)'' workshop, a workshop series begun in
2015 to facilitate communication regarding R\&D on ML applications in
HEP.\\

\noindent
\textbf{HEP Analysis Ecosystem Retreat}\\
\emph{Date:} 22-24 May, 2017\\
\emph{Location:} Amsterdam, the Netherlands\\
\emph{URL:}
\href{http://indico.cern.ch/event/613842/}{{http://indico.cern.ch/event/613842/}}\\
\emph{Summary report:}
\href{http://hepsoftwarefoundation.org/assets/AnalysisEcosystemReport20170804.pdf}{{http://cern.ch/go/mT8w}}\\
\emph{Description:} This was a general workshop, organised about the
HSF, about the ecosystem of analysis tools used in HEP and the ROOT
software framework. The workshop focused both on the current status and
the 5-10 year time scale covered by the CWP.\\

\noindent
\textbf{CWP Event Processing Frameworks Workshop}\\
\emph{Date:} 5-6 Jun, 2017\\
\emph{Location:} FNAL (Batavia, IL, USA)\\
\emph{URL:}
\href{https://indico.fnal.gov/conferenceDisplay.py?confId=14186}{{https://indico.fnal.gov/conferenceDisplay.py?confId=14186}}\\
\emph{Description:} This was a workshop held by the Event Processing Frameworks
CWP working group focused on writing an initial draft of the framework
white paper. Representatives from most of the current practice
frameworks participated.\\

\noindent
\textbf{HEP Software Foundation Workshop}\\
\emph{Date:} 26-30 Jun, 2017\\
\emph{Location:} LAPP (Annecy, France)\\
\emph{URL:}
\href{https://indico.cern.ch/event/613093/}{{https://indico.cern.ch/event/613093/}}\\
\emph{Description:} This was the final general workshop for the CWP
process. The CWP working groups came together to present their status
and plans, and develop consensus on the organisation and context for the
community roadmap. Plans were also made for the CWP writing phase that
followed in the few months following this last workshop.

\newpage
\hypertarget{appendix-b-glossary}{%
\section{Glossary}\label{appendix-b-glossary}}

\begin{description}

\item[AOD] Analysis Object Data is a summary of the reconstructed event and
contains sufficient information for common physics analyses.

\item[ALPGEN] An event generator designed for the generation of Standard Model
processes in hadronic collisions, with emphasis on final states with
large jet multiplicities. It is based on the exact LO evaluation of
partonic matrix elements, as well as top quark and gauge boson decays
with helicity correlations.

\item[BSM] Physics beyond the Standard Model (BSM) refers to the theoretical
developments needed to explain the deficiencies of the Standard Model
(SM), such as the
\href{https://en.wikipedia.org/wiki/Origin_of_mass}{origin of mass}, the
\href{https://en.wikipedia.org/wiki/Strong_CP_problem}{strong CP
problem},
\href{https://en.wikipedia.org/wiki/Neutrino_oscillation}{neutrino
oscillations},
\href{https://en.wikipedia.org/wiki/Baryon_asymmetry}{matter--antimatter
asymmetry}, and the nature of
\href{https://en.wikipedia.org/wiki/Dark_matter}{dark matter} and
\href{https://en.wikipedia.org/wiki/Dark_energy}{dark energy}.

\item[Coin3D] A C++ object oriented retained mode 3D graphics API
used to provide a higher layer of programming for OpenGL.

\item[COOL] LHC Conditions Database Project, a subproject of the POOL persistency framework.

\item[Concurrency Forum] Software engineering is moving towards a paradigm shift in order to accommodate new CPU
architectures with many cores, in which concurrency will play a more fundamental role in programming languages
and libraries. The forum on concurrent programming models and frameworks aims to share knowledge among interested
parties that work together to develop 'demonstrators' and agree on technology so that they can share code and compare results.

\item[CRSG] Computing Resources Scrutiny Group, a WLCG committee in charge of scrutinizing and assessing LHC experiment yearly resource requests to prepare funding agency decisions.

\item[CSIRT] Computer Security Incident Response Team. A CSIRT provides a reliable and trusted single point of contact for reporting computer security incidents and taking the appropriate measures in response tothem.

\item[CVMFS] The CERN Virtual Machine File System is a network file system
based on HTTP and optimised to deliver experiment software in a fast,
scalable, and reliable way through sophisticated caching strategies.

\item[CWP] The Community White Paper (this document) is the result of an
organised effort to describe the community strategy and a roadmap for
software and computing R\&D in HEP for the 2020s. This activity is
organised under the umbrella of the HSF.

\item[Deep Learning (DL)] one class of Machine Learning algorithms, based on a
high number of neural network layers.

\item[DNN] Deep Neural Network, class of neural networks with typically a
large number of hidden layers through which data is processed.

\item[DPHEP] The Data Preservation in HEP project is a collaboration for data
preservation and long term analysis.

\item[EGI] European Grid Initiative. A European organisation in charge of
delivering advanced computing services to support scientists,
multinational projects and research infrastructures, partially funded by
the European Union. It is operating both a grid infrastructure (many
WLCG sites in Europe are also EGI sites) and a federated cloud
infrastructure. It is also responsible for security incident response
for these infrastructures (CSIRT).

\item[FAIR] The Facility for Antiproton and Ion Research (FAIR) is located at
GSI Darmstadt. It is an international accelerator facility for research
with antiprotons and ions.

\item[FAIR] An abbreviation for a set of desirable data properties: Findable,
Accessible, Interoperable, and Re-usable.

\item[FCC] Future Circular Collider, a proposed new accelerator complex for
CERN, presently under study.

\item[FCC-hh] A 100 TeV proton-proton collider version of the FCC (the ``h''
stands for ``hadron'').

\item[GAN] Generative Adversarial Networks are a class of
\href{https://en.wikipedia.org/wiki/Artificial_intelligence}{artificial
intelligence} algorithms used in
\href{https://en.wikipedia.org/wiki/Unsupervised_machine_learning}{unsupervised
machine learning}, implemented by a system of two
\href{https://en.wikipedia.org/wiki/Neural_network}{neural networks}
contesting with each other in a
\href{https://en.wikipedia.org/wiki/Zero-sum_game}{zero-sum game}
framework.

\item[Geant4] A toolkit for the simulation of the passage of
particles through matter.

\item[GeantV] An R\&D project that aims to fully exploit the
parallelism, which is increasingly offered by the new generations of CPUs, in the field of detector simulation.

\item[GPGPU] General-Purpose computing on Graphics Processing Units is the use of a Graphics Processing Unit (GPU), which typically handles computation only for computer graphics, to perform computation in applications traditionally handled by the Central Processing Unit (CPU). Programming for GPUs is typically more challenging, but can offer significant gains in arithmetic throughput.

\item[HEPData] The Durham High Energy Physics Database is an open access
repository for scattering data from experimental particle physics.

\item[HERWIG] This is an event generator containing a wide range of Standard
Model, Higgs and supersymmetric processes. It uses the parton-shower approach
for initial- and final-state QCD radiation, including colour coherence effects and
azimuthal correlations both within and between jets.

\item[HL-LHC] The High Luminosity Large Hadron Collider is a proposed upgrade to
the Large Hadron Collider to be made in 2026. The upgrade aims at increasing the
luminosity of the machine by a factor of 10, up to
$10^{35}\mathrm{cm}^{-2}\mathrm{s}^{-1}$,
providing a better chance to see rare processes and improving
statistically marginal measurements.

\item[HLT] High Level Trigger. The computing resources, generally a large farm, close to the detector which process the events in real-time and select those who must be stored for further analysis.

\item[HPC] High Performance Computing.

\item[HS06] HEP-wide benchmark for measuring CPU performance based on the SPEC2006 benchmark
(\href{https://www.spec.org}{{https://www.spec.org}}).

\item[HSF] The HEP Software Foundation facilitates coordination and common
efforts in high energy physics (HEP) software and computing
internationally.

\item[IML] The Inter-experimental LHC Machine Learning (IML) Working Group is
focused on the development of modern state-of-the art machine learning
methods, techniques and practices for high-energy physics problems.

\item[IOV] Interval Of Validity, the period of time for which a specific piece
of conditions data is valid.

\item[JavaScript] A high-level, dynamic, weakly typed,
prototype-based, multi-paradigm, and interpreted programming language.
Alongside HTML and CSS, JavaScript is one of the three core technologies
of World Wide Web content production.

\item[Jupyter Notebook] This is a server-client application that allows
editing and running notebook documents via a web browser. Notebooks are
documents produced by the Jupyter Notebook App, which contain both
computer code (e.g., python) and rich text elements (paragraph,
equations, figures, links, etc...). Notebook documents are both
human-readable documents containing the analysis description and the
results (figures, tables, etc..) as well as executable documents which
can be run to perform data analysis.

\item[LHC] Large Hadron Collider, the main particle accelerator at CERN.

\item[LHCONE] A set of network circuits, managed worldwide by the National
Research and Education Networks, to provide dedicated transfer paths for
LHC T1/T2/T3 sites on the standard academic and research physical
network infrastructure.

\item[LHCOPN] LHC Optical Private Network. It is the private physical and IP
network that connects the Tier0 and the Tier1 sites of the WLCG.

\item[MADEVENT] This is a multi-purpose tree-level event generator. It is
powered by the matrix element event generator MADGRAPH, which generates
the amplitudes for all relevant sub-processes and produces the mappings
for the integration over the phase space.

\item[Matplotlib] This is a Python 2D plotting library that provides
publication quality figures in a variety of hardcopy formats and
interactive environments across platforms.

\item[ML] Machine learning is a field of computer science that gives computers
the ability to learn without being explicitly programmed. It focuses on
prediction making through the use of computers and emcompasses a lot of
algorithm classes (boosted decision trees, neural networks\ldots{}).

\item[MONARC] A model of large scale distributed computing based on many
regional centers, with a focus on LHC experiments at CERN. As part of
the MONARC project, a simulation framework was developed that provides a
design and optimisation tool. The MONARC model has been the initial
reference for building the WLCG infrastructure and to organise the data
transfers around it.

\item[OpenGL] Open Graphics Library is a cross-language, cross-platform
application programming interface(API) for rendering 2D and 3D vector
graphics. The API is typically used to interact with a graphics
processing unit(GPU), to achieve hardware-accelerated rendering.

\item[Openlab] CERN openlab is a public-private partnership that accelerates
the development of cutting-edge solutions for the worldwide LHC
community and wider scientific research.

\item[P5] The Particle Physics Project Prioritization Panel is a scientific
advisory panel tasked with recommending plans for U.S. investment in
particle physics research over the next ten years.

\item[PRNG] A PseudoRandom Number Generator is an algorithm for generating a
sequence of numbers whose properties approximate the properties of
sequences of random numbers.

\item[PyROOT] A \href{http://www.python.org/}{Python} extension module that
allows the user to interact with any ROOT class from the Python
interpreter.

\item[PYTHIA] A program for the generation of high-energy physics events, i.e.,
for the description of collisions at high energies between elementary
particles such as e+, e-, p and pbar in various combinations. It
contains theory and models for a number of physics aspects, including
hard and soft interactions, parton distributions, initial- and
final-state parton showers, multiparton interactions, fragmentation and
decay.

\item[QCD] Quantum Chromodynamics, the theory describing the strong
interaction between quarks and gluons.

\item[REST] Representational State Transfer
\href{https://en.wikipedia.org/wiki/Web_service}{web services} are a way
of providing interoperability between computer systems on the Internet.
One of its main features is stateless interactions between clients and
servers (every interaction is totally independent of the others),
allowing for very efficient caching.

\item[ROOT] A modular scientific software framework widely used in HEP data
processing applications.

\item[SAML] Security Assertion Markup Language. It is an open, XML-based,
standard for exchanging authentication and authorisation data between
parties, in particular, between an identity provider and a service
provider.

\item[SDN] Software-defined networking is an umbrella term encompassing several kinds of network technology aimed at making the network as agile
and flexible as the virtualised server and storage infrastructure of the modern data center.

\item[SHERPA] Sherpa is a Monte Carlo event generator for the Simulation of
High-Energy Reactions of PArticles in lepton-lepton, lepton-photon,
photon-photon, lepton-hadron and hadron-hadron collisions.

\item[SIMD] Single instruction, multiple data (\textbf{SIMD}), describes
computers with multiple processing elements that perform the same
operation on multiple data points simultaneously.

\item[SM] The Standard Model is the name given in the 1970s to a theory of
fundamental particles and how they interact. It is the currently
dominant theory explaining the elementary particles and their dynamics.

\item[SWAN] Service for Web based ANalysis is a platform for interactive data
mining in the CERN cloud using the Jupyter notebook interface.

\item[TBB] Intel Threading Building Blocks is a widely used C++ template
library for task parallelism. It lets you easily write parallel C++
programs that take full advantage of multicore performance.

\item[TMVA] The Toolkit for Multivariate Data Analysis with ROOT is a
standalone project that provides a ROOT-integrated machine learning
environment for the processing and parallel evaluation of sophisticated
multivariate classification techniques.

\item[VecGeom] The vectorised geometry library for particle-detector simulation.

\item[VO] Virtual Organisation. A group of users sharing a common interest
(for example, each LHC experiment is a VO), centrally managed, and used
in particular as the basis for authorisations in the WLCG
infrastructure.

\item[WebGL] The Web Graphics Library is a JavaScript API for rendering
interactive 2D and 3D graphics within any compatible web browser without
the use of plug-ins.

\item[WLCG] The Worldwide LHC Computing Grid project is a global collaboration
of more than 170 computing centres in 42 countries, linking up national
and international grid infrastructures. The mission of the WLCG project
is to provide global computing resources to store, distribute and
analyse data generated by the Large Hadron Collider (LHC) at CERN.

\item[X.509] A cryptographic standard which defines how to implement service
security using electronic certificates, based on the use of a private
and public key combination. It is widely used on web servers accessed
using the https protocol and is the main authentication mechanism on the
WLCG infrastructure.

\item[x86\_64] 64-bit version of the x86 instruction set.

\item[XRootD] Software framework that is a fully generic suite for fast, low latency and scalable data access.

\end{description}

\end{appendices}
