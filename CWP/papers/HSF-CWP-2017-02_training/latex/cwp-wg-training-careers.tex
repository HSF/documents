% Copyright (C) 2018, The HSF Community White Paper authors, licence CC-BY-4.0.

% JHEP preprint template
\documentclass[12pt,a4paper]{article}
\usepackage{jheppub}

\usepackage{graphicx}
\usepackage{subcaption}
\usepackage{xspace}

\usepackage[utf8]{inputenc}

\usepackage[style=numeric-comp,sorting=none]{biblatex}
\addbibresource{../../HSF-CWP-2017-01_roadmap/latex/cwp.bib}
\addbibresource{../../HSF-CWP-2017-01_roadmap/latex/cwp-chapters.bib}

\abstract{The rapidly increasing evolution of technology and the parallel increasing complexity of algorithmic analysis in HEP (and in science in general) require a much larger portfolio of specific skills by the developers with respect to the past. Young researchers, graduating from universities worldwide, currently do not receive adequate preparation in the very diverse fields of modern computing to respond to growing necessities posed by the most advanced experimental challenges. The vast majority of the HEP community strongly agrees on the need for an overarching umbrella of training programs to bring young researchers (but also more mature ones) up to date with new software technologies, such as, for example, concurrent programming and artificial intelligence. }


\begin{document}

\begin{tabular*}{\linewidth}{lc@{\extracolsep{\fill}}r@{\extracolsep{0pt}}}
 & & HSF-CWP-2017-02 \\
 & & June 6, 2018 \\ % use \date or hardwire e.g. December 15, 2017
 & & \\
\end{tabular*}
\vspace{2.0cm}

\title{HEP Software Foundation Community White Paper Working Group --
Training, Staffing and Careers}

%\linenumbers

\author[1]{Dario Berzano,}
\author[2]{Riccardo Maria Bianchi,}
\author[3]{Peter Elmer,}
\author[4]{Roger Jones,}
\author[5]{Michel Jouvin,}
\author[6]{Daniel S. Katz,}
\author[7]{Sudhir Malik,}
\author[8]{Dario Menasce,}
\author[6]{Mark Neubauer,}
\author[9,a]{Albert Puig,}
\author[1]{Graeme Stewart,}
\author[10]{and Christopher Tunnell}

\affiliation[1]{CERN, Geneva, Switzerland}
\affiliation[2]{Department of Physics and Astronomy, University of Pittsburgh, Pittsburgh, PA, USA}
\affiliation[3]{Princeton University, Princeton, NJ, USA}
\affiliation[4]{Lancaster University, Lancaster, UK}
\affiliation[5]{LAL, Université Paris-Sud and CNRS/IN2P3, Orsay, France}
\affiliation[6]{University of Illinois Urbana-Champaign, Urbana, IL, USA}
\affiliation[7]{University of Puerto Rico, Mayaguez, PR, USA}
\affiliation[8]{INFN Sezione di Milano-Bicocca, Italy}
\affiliation[9]{Physik-Institut, Universität Zürich, Zürich, Switzerland}
\affiliation[10]{University of Chicago, Chicago, IL, USA}
\affiliation[a]{Supported by SNF under contract 168169.}

\maketitle

\tableofcontents
\clearpage

%-------------------------------------------------------------------------
%\newpage
%\section{Training the community  \textcolor{pink}{Will move into a NEW chapter} }
%\label{sec:training}

% ---------------------------------------------------
% CHAPTER SUMMARY MESSAGE: Motivate the need for training in the HEP community,
%  summarize existing training efforts to motivate that we already spend funding on
% this. Discuss initiatives and programs which can be developed in the coming years
% including a discussion of the known training needs of the community. 
%
%    Goals for this chapter 
%   - Motivation to dedicate resources to training
%   - The community does not recognize the importance
%   - people learn on the job
%   - are there objective criteria
%   - retaining people with skills? 
%
%   see e.g. https://github.com/particleist/inventingcarrots/blob/master/inventing_carrots_merged.md which tries to address the problem of "how do we incentivize affirm and reward learning about software in HEP"
%
%   - People should have incentives to 
%
% ---------------------------------------------------

%Contributors: Fernanda Psihas, Justin Vasel, Michela Paganini, Douglas Davis, Savannah
%Thais, Martin Vala, Jonas Eschle, Xavier Vilasís-Cardona, Jamal Rorie, Sean-Jiun Wang,
%Liz Sexton-Kennedy, Sergei Gleyzer, Laurent Basara, Kurt Rinnert, Bob Stienen, Matt Bellis, Meghan Kane (Industry - SoundCloud), Dario Menasce, Pat Skubic, Hadrien Grasland, Riccardo Maria Bianchi, Daniel S. Katz, Sudhir Malik, Albert Puig


%____________________________________________________________________
%____________________________________________________________________
\section{Introduction}

Throughout the last few decades, the field of high energy physics (HEP) has
consistently either driven or utilized the latest innovations in computational
tools and technologies for problem solving.
The toolset of the particle physicist is vast and ever-growing, and the problem
set increasingly complex.
A typical graduate student or postdoc working on HEP experiments will likely
encounter a number of tools, software, and programming languages with which they
are totally unfamiliar.
This requires ``on-the-job'' training for the same people whose work drives the
reach of our analyses, while they are developing the tools for these analyses.

A strong research community should consist of independent actors who ask
research questions and direct their own analyses, instead of blindly following
prescriptions and recipes. It is therefore imperative that the community be
proactive about providing training resources and opportunities for the good of
the community, and that we follow through on the promise that adequate career
path possibilities exist for people in HEP who would otherwise leave the field
due to limited advancement opportunities.

To produce the best possible science, it is important, for the broad HEP
community, to invest in training physicists in essential software and software
engineering skills, which exist at many different levels. The level of depth of
software engineering competences in which a physicist should be trained will
vary depending on the problems that they face, and the appropriate level may
evolve over time according to the problem at hand. For example, for a researcher
who only performs basic data analysis, it may be sufficient to be familiar with
just Python and data analysis libraries. However, contributing to a large open
source project, such as ROOT, requires greater training in software development
and related practices. These higher-order skills may include an understanding of
writing testable code, time and space complexity implications, and knowledge of
common software architecture paradigms. The need for these skills cannot be
fulfilled completely just by hiring specialists in computer science, since a
thorough understanding of the physics problems is usually very important to the
success of the effort, making it necessary to provide a career path for those
young physicists willing to steer their interests in that particular direction.

In each HSF workshop, a discussion on the need for proper tutoring, training,
and the relevant career possibilities in HEP computing emerged and was felt to
be crucial and essential by almost all participants. There is a strong consensus
that future software developments will benefit enormously from a successful
implementation of such training programs, provided proper career prospects are
put in place, as an incentive, to would-be trainers.

This document tries to address the most relevant questions related to the
training in the HEP community, namely:
\begin{itemize}
    \item How to encourage and provide incentives to the members of the HEP 
    community to train students and/or other collaborators?
    \item How to properly assign credit to software development (and training 
    thereof) activities?
    \item What are the current practices in the field regarding hiring of 
    researchers with a strong specialisation in computing and the policies 
    to retain such an expertise in the long term? Is there room for improvement?
    \item How to address and bridge the gap in formal software training that 
    is not always imparted by universities?
\end{itemize}

%____________________________________________________________________
%____________________________________________________________________
\section{Motivation}

%    DOES THIS SECTION ANSWER THE FOLLOWING: 
%    - How is training different in software? Specify that there is /no formal training for these tools? 

The introduction and development of increasingly complex tools has led to
increasing improvements in the performance of our analyses. However, the
personnel who drive these advancements through novel implementations of the
technology are the analyzers. In this context, the term {\it analyzers} includes
all those researchers who participate in an HEP experiment whose contribution
includes designing and implementing a high speed DAQ system, setting up a modern
database to store the state of a detector, complex task of managing the data
flow from the first trigger level to the final n-tuple dataset and creating
sophisticated statistical and mathematical tools to extract publishable physical
meaning from those same data.

The development of new tools is insufficient to drive advances in the field
unless researchers who have the expertise apply them to physics questions.
Furthermore, a larger and a better-connected community, with expertise in the
tools and access to the new developments, can lead to an open-source model in
HEP. The efforts to train the community in all relevant tools is, thus, a
priority within the R\&D effort itself.

However, as tools become more complex, the learning rate of developers must keep
up if we are to utilize them effectively.  At the same time, as in other
branches of research, it is essential to sustain efforts in maintaining
reproducibility of the usage of the tools. This again highlights the need for
physicts to be appropriately trained in computing science, rather than
professional computing scientists, whose goals and interests may not necessarily
be aligned or focused on the physics results of a particular HEP experiment.

To avoid duplicate efforts within the community, it is important to find a way
to approach training efforts holistically and cross-experimentally, whenever
possible. This requires a certain level of organization and coordination in
order to guarantee the delivery of efficient, non-redundant software to
different experiments and to allow the different communities to agree on common
standards and procedures.

Beyond the immediate benefit to the scientific community in having well-trained
collaborators, most people who start a physics graduate program will have
careers outside of academia. This trend is reflected in several calls for
proposals from national scientific funding agencies like H2020 and those from
the National Science Foundation (NSF) that promote training in computer
infrastructure and information areas.

%\textcolor{red}{this call is more to develop cyberinfrastructure staff than scientists - for example, people at computing centers}

\begin{quote}
\it The overarching goal of these programs is to prepare, nurture and grow the national scientific workforce for creating, utilizing, and supporting advanced cyberinfrastructure (CI) that enables cutting-edge science and engineering and contributes to the Nation's overall economic competiveness and security. 

{\bf - Training-based Workforce Development for Advanced Cyberinfrastructure (CyberTraining) Webinar}\footnote{ \url{https://www.nsf.gov/events/event_summ.jsp?cntn_id=190179&org=NSF}}
\end{quote}


\begin{quote}
\it New information, communication, and computational technologies have had profound impacts on the practice of science (in this solicitation, the term science includes the natural, mathematical, computing, and social sciences),  engineering, and education. This includes the means by which citizens of all ages use science  and engineering to enhance professional and private lives.  The systems, tools, and services emerging from these new technologies are linked to create a comprehensive cyberinfrastructure that is enabling individuals, groups, and organizations to advance research and education in ways that revolutionize who can participate, what they can do, and how they do it. Sustaining this revolution across all areas of science, engineering, and education requires the formation of a citizenry and workforce with the knowledge and skills needed to design and deploy as well as adopt and apply these cyber-based systems, tools and services over the long-term. The opportunity for such preparation should be available at all stages of formal and informal education (K-16 and lifelong), training and professional development, and must be extended to all individuals and communities. 

{\bf - Cyberinfrastructure Training, Education, Advancement, and Mentoring for Our 21st Century Workforce (CI-TEAM)}\footnote{ \url{https://www.nsf.gov/funding/pgm_summ.jsp?pims_id=12782}}
\end{quote}

An open issue is the kind of incentive that our community can collectively
leverage in order to increase the number of people (with strong expertise in
computing) who decide to dedicate time and effort to train fellow members of
their community.
To be effective, the training must be done by people who work at the cutting
edge of the technology, who in many cases are found among the youngest
researchers. These are also the very same people who are most in need of
officially recognized credits for the advancement of their careers. The current
practice in HEP is that these young scientists only receive credit and
recognition (in the vast majority of cases) by working on a data analysis
project, and this discourages them from finding time to transfer their acquired
know-how to others in an effective way. A possible solution to this conundrum
could be the establishment of specific career paths for researchers with a
computer science specialization. The practical implementation of such an
approach, however, is extremely challenging as there are a wide range of
institutions involved that belong to different countries with their own
policies, priorities, and funding strategies. Broadly, two career paths could be
envisaged. Both would recognise that domain-sepcific knowledge and continued
engagement is vital to the success of HEP computing and software. While in an
ideal world (for physicists, at least), a physicst could make a detailed plan of
what they need (expressed in a tidy requirements document) and a computer
scientist could use this document to provide the required solution, with very
little additional exchange of information during the process (which is how the
software development process sometimes works in industry). In the real HEP
world, instead, most of the time the requirements are bound to change rather
frequently for a variety of reasons (not only because of poor planning) forcing
developers and physicsts to closely interact during the entire process of
software development; this works smoothly only when both communities have the
same goals and speak the same jargon, hence the need for physicists with a
computer science specialisation.

The first possible career would be the path of a \emph{physicist researcher with
computing science specialisation}, in some places corresponding to the role of a
physicist-programmer. It would be conceptually different to that of a
\emph{computer scientist}: the two have different goals, seek different paths to
the solution of their problems, and usually do not even adopt the same jargon in
expressing problems and solutions. Such a role would have an active role in
physics analysis, and so meet many of the traditional metrics for an academic
career path.

The second would be the path of a domain-specific \emph{researcher software
engineer}. Such a person has the primary role of developing software and finding
software solutions, but with a large domain-specific knowledge to both
understand the use cases and clients, but also of the available solutions.  Such
posts are almost impossible to develop in the current academic frameworks; but
while RSEs are understood by many funding agencies and universities as a
technical post, they are seen as available for short-term engagements with a
researcher community, delivering a quick fix or a refactoring and then moving
one. This does not deliver the domain specific expertise required.

Both career paths suffer in academic terms from few or poorly cited
publications, few opportunities to win grants in their own right and little
opportunity for impact with commerce and wider society. These three failings
make an academic appointment challenging or disfavoured compared to other
applicants. However, there are measures that an be taken to address all three
weaknesses.

% This makes the development of specific solutions even more difficult and
% cumbersome.

%Finally, the career of these two figures follows slightly different paths, with different requirements. 
%____________________________________________________________________
%\subsection{Existing Training efforts  \textcolor{pink}{is this needed?....}}
% why not, shows what is done already


%____________________________________________________________________
%____________________________________________________________________
\section{Training needs of the community}

% DOES THIS SECTION ANSWER THE FOLLOWING: 
% - Different levels of training are required 

The HEP user community consists of a set of users with diverse software
experience, interests, and time availablity to learn new techniques.  Any
training program must take into account the variation in target audiences. For
example, an undergraduate doing a summer research project has different needs
and skills than their mentoring professor.
For purposes of training, it is useful to think of all our user community as
\emph{students}. More specifically, we can broadly identify four different sets
of students, classified by experience level, recognizing that even within these
groups there will be people who will learn at different speeds and with
differing depths of understanding:

\begin{itemize}
   \item \emph{Beginners}: New collaborators with no knowledge of the 
   tools or techniques they are expected to use. These might be undergraduates 
   but they could also be senior researchers who have not yet or not recently 
   engaged in software analysis. These are people in need of some kind of formal 
   training in modern computing techniques (basically compiled and scripting 
   languages together with operating system basics such as filesystems, version 
   control, and command-line shells).
   \item \emph{Intermediate users}: HEP analyzers with some experience in 
   either analysis concepts and/or tools, but  looking to supplement their 
   experiences with more recent and modern approaches. 
   \item \emph{Advanced users}: HEP analysis experts who have mastered current 
   technologies and implementations (at a certain point in time) and want to 
   stay up-to-date with new advanced developments.
   \item \emph{Software developers}: HEP scientists in charge of software 
   development in areas not limited only to analysis, such as DAQ systems, 
   computing infrastructure, databases, pattern recongition, and complete 
   frameworks.
\end{itemize}

Each of these groups has vastly different training needs. However,
\emph{whenever possible}, any training program should take advantage of
developments in pedagogy, such as active
learning\footnote{\url{http://www.crlt.umich.edu/tstrategies/tsal}}, peer
learning\footnote{\url{https://en.wikipedia.org/wiki/Peer_learning}}, and
web-based training.\footnote{See \S\ref{sec:initiatives}.} In some cases, it may
even be advantageous to hand out code samples that are purposely broken or
flawed, and ask students to fix or improve them. Learning the material in a way
that sticks with the students is difficult and challenging for both the students
and the instructor and often takes more time than we would prefer. However, this
is the best way to ensure an educated community of scientists who can fully
contribute to the physics programs at large, which is really the ultimate goal
of any training program.

%____________________________________________________________________
\subsection{Knowledge that needs to be transferred}

At all stages of software and computing training, we should take care to
encourage \emph{Good Practices Across the Community} (GPAC), such as error
checking, modularity of code design, version control, writing tests, etc. All
the key concepts addressed in the training should not be specific to a
particular experiment or field of application, but general enough to be useful
to the whole HEP community and possibly beyond. In this section, we present a
list of specific concepts that need to be taught to members of the community, in
order to guarantee the base level of competence needed to write efficient code
for the different tasks performed in HEP experiments.
 
An essential knowledge base that needs to be transferred includes basic
programming concepts, data structures, basics of code design, error checking,
code management tools, validation and debugging tools. More advanced topics
include modularity of code design, advanced data structures, evaluation metrics,
writing tests and working with different types of hardware accelerators.
Cutting-edge approaches such as the use of hardware accelerators (whether GPUs
or FPGAs), Deep Neural Networks, and Machine Learning should be considered
important training topics. Additionally, special emphasis should be made on
reporting results and documenting them.
The training subjects that the community considers important to pursue are
listed as follows:

\begin{itemize}
   \item Basic and Advanced Programming Concepts
   \begin{itemize}
      \item Object oriented paradigm
      \item Compiled langauges (C++)
      \item Scripting languages (Python, Javascript,...)
   \end{itemize}
   \item Algorithmics 
   \begin{itemize}
      \item Boost library
      \item STL algorithms for containers
      \item R and/or ROOT
   \end{itemize}
   \item Existing frameworks (development or application level)
   \begin{itemize}
      \item Qt
      \item ROOT
      \item experiment specific framework (possibily if of potential interest outside the native experiment)
   \end{itemize}
   \item Code design (design patterns)
   \item Development tools
   \begin{itemize}
      \item IDEs (Integrated Development Environment)
      \item Debuggers
      \item Profilers
   \end{itemize}
   \item Evaluation metrics
   \item “Trust” metrics such as data driven tests
   \item Specific software implementation training
   \item Good practices 
   \item Code style and clarity
   \item Scripting and data cleaning
   \item Reporting results reproducibly
   \item Writing Documentation
\end{itemize}


%____________________________________________________________________
%____________________________________________________________________
\section{Challenges}

Within a single experiment, different skill sets are needed. In addition to a
skill set that contains basic programming language knowledge, testing and code
management tools, and experiment-specific framework knowledge, there are more
specialized skills that only a subset of the community needs to know, such as
software optimization, or low-level hardware interfaces.
While there already are experiment-specific training efforts in place, there are
many needs that are in common between experiments. Therefore, we should probably
strive to extract the common knowledge and build a common training program. This
would reduce duplication of efforts and enable growth of a shared training
culture by accumulating and sharing expertise across different experiments.

To realise this goal, a possibility could be the institution of a {\em
federation} of existing training initiatives, a very lightwheight umbrella just
aimed at improving the efficiency and cost-effectivness of this important
activity.

\section{Implementation Roadmap}
The implementation of training should target multiple training formats and knowledge transfer, such as videos, wikis, lectures, jupyter notebooks and advanced visualizations, etc.
People who learn in non-conventional ways should not be excluded from the
opportunity to learn. Since there are different stages of knowledge transfer,
programs should be developed that encourage experts to share knowledge. Issues
considered in this section include global training initiatives, strategies for
leveraging existing forums, and training tool development.

An important point to consider is the difficulty, often experienced in the past,
in developing large software programs across different experiment communities.
It is always challenging to organize a Working Group for training across
experiments or even across diverse communities (such as HEP and Astroparticle,
for example). This is a point where an appropriate incentive program to reward
those who train the community might help.

%____________________________________________________________________
\subsection{Initiatives for future training programs\label{sec:initiatives}} 

Some methods that can be used for global training include Massive Online Open
Courses, hands-on workshops, online knowledge bases, expert trainer volunteer
networks, and web-based training approaches.

Massive Online Open Courses (MOOCs) can be used to develop an open-source set of
tutorials and tools. Existing online courses such as Udacity and Coursera can be
evaluated and recommended to the community. A novel approach such as WikiToLearn
could also explored to assess potential benefits (as has already been attempted
in the context of the GridKa School of Computing, see below).

Experiment-specific and global knowledge bases can be established with
incentives for experts to contribute. They can be open-source so that a lot of
knowledge can be added by the trainees themselves as information is learned
(this is the particular context where an approach such as WikiToLearn could be
of great help).

Question-and-answer websites such as Stack Overflow for HEP are also a very
useful resource for common problems and questions. This approach has already
been considered by HSF start-up members, and turned out to be difficult to
pursue for a number of reasons, but in the future, boundary conditions might
change, making this approach viable.

Hands-on workshops are an invaluable part of learning how to apply theoretical
concepts in practice. Identifying which of the current workshops are productive
and useful, and if they cover all the topics that need to be transferred and
that are in demand by the students, is an important action item.

Creating an expert tutor volunteer network is another way to provide training
and support to the community.
This of course requires proper recognition, at least in terms of career
prospects as an incentive for young researchers. It is clear that the best
possible tutors are, in principle, those people who are actively engaged in
modern software developments: these are in general young researchers in the
first steps of their careers and, in order to be attracted to devoting
substantial time to training and tutoring, they must be assured that such an
effort would be properly recognised in an official way. Such kind of recognition
is not currently implemented, at least not in a standardized or official way. A
possible structure for such a network could be the establishment of a {\em
federation} of existing schools.

It should be considered that some professors, who act as graduate student
advisors, might need to be encouraged to make sure their students are properly
trained. Sometimes, students are instead pushed to learn the bare minimum to get
the work done, at the expense of a broader training/education curriculum that
could actually yield improved results, but further down the line. One incentive
would be to provide training programs that also count as course credit, perhaps
as an elective. This model is already in limited use with some online
solutions\footnote{\url{https://www.edx.org/credit}}, but this is not universal.
Discussions should be started with collaborators at higher education
institutions to see what the roadblocks or opportunities would be for these
training sessions to serve double duty.
It should also be considered that not all students will end their career in
research or academia: their contribution to the research activity, as students,
should therefore also provide them knowledge, know-how, and skills considered a
valuable asset by industry, in order to increase their chances of a career
outside research.

Difficulties that have emerged in the past with respect to implementing training
courses are the lack of funding and the lack of available time by experts in the
field. People with enough expertise or insight in the field usually don't have
time to devote to prolonged periods of student training, and, even when they can
find time, the cost of setting up a training course in an effective way is often
beyond what's made available by funding agencies (funds for travel, hosting,
setting up a room with a computing infrastructure to allow interactive hands-on
session, etc.) A possible solution is a completely different approach to
training (but complementary to the already existing and successful classical
efforts such as the CERN School of Computing's Bertinoro and Kit ones): instead
of directly teaching to students, trainees could make use of a web-based
platform to provide training materials to students. This complementary approach
has several advantages over traditional ones:
\begin{itemize}
   \item Tutors can add material to the web site at a very slow pace (whenever
   they find time to do it, one slide a week or a chapter a day) \item Their
   material, made public on a web-site, can be further expanded by collaborators
   (also at their own pace) or, even better and more productively, by students
   who decide to contribute new additions, examples, exercises etc. Such a
   collaborative effort allows more people to be exposed to training at any
   given time, creates a sense of community, and creates bridges between people
   in contiguous areas or research. Students can use the same platform to
   exchange their own examples, make suggestions and point out interesting
   concepts. In such a model, the possible contribution from others to the
   training material needs to be moderated and validated with appropriate
   policies.
   \item If complemented by the availability of remote virtual machines
   (possibly through a browser), students could have access to examples and
   exercises that are already embedded in their own natural environment: all the
   necessary tools/libraries needed to implement the exercise will be already
   available in the virtual machine (possibly a Docker container). With just a
   web browser, students could run complex examples from home, taking advantage
   of a remote facility that provides some storage and computing power.
   Important here is the concept of ``environment'': a Docker container could be
   set up in such a way that students will work in an exact replica of the
   environment they will be exposed to in their experiment (or collaboration at
   large). Moreover, students could be provided with Docker containers that
   preserves their modified environment across sessions, allowing them to
   develop their skills over a prolonged period of time by accessing all the
   files that were made persistent day by day during the training.
   \item There would no longer be a need to find the resources to host a school
   and pay the tutor(s) (and eventually subsidize the students to participate in
   training in a remote location). Students could follow the training at their
   own pace from wherever they happen to be. A traditional school only lasts for
   5 days (10 at most) and it is difficult to cover a  subject to any
   significant depth in such a short time. The web approach, instead, would
   allow for very long and in-depth coverage of any kind of subject, and in this
   sense it could be a {\it complementary} approach to a traditional school. Of
   particular interest could be courses such as, e.g., ``Machine Learning'',
   ``Statistical Analyis with ROOT'', or even just ``Good practices in C++'' or
   ``Python Programming for scientific computing''.
   \item Finally, this approach could allow the creation of {\it browsable}
   repositories of all training materials, grouped by argument, by relevance, by
   experiment or whatever other critieria. Students from all over the world
   could be exposed to a large repository of examples, exercises and in general
   training material from their own home.
\end{itemize}

An example of a such a web-based platform already exists, and has been
implemented as an Open Source project (backed by Wikimedia) by a group of more
than 30 Italian students. The project is WikiToLearn
\footnote{\url{https://it.wikitolearn.org/}}. It hosts training material in
several languages, for several disciplines, ranging from Economics to Physics,
Mathematics, and several others. Because it is based upon
wikimedia\footnote{\url{https://www.wikimedia.org/}} software, it is very easy
to add material to the site, and to make it appear under a specific topic (such
as, e.g., Software/Techniques/Machine-Learning) and to manipulate it as if it
were a single document. In the end, students can selectively choose individual
chapters from the site and have the corresponding pdf sent them as a book,
complete with index, content, and chapters.

The adoption of such an approach is made rather easy in WikiToLearn by the
relative simplicity of the wikimedia-based toolset: users contribute their
training material using just a web-browser, and in order to do this efficiently,
the learning curve has been kept appropriately shallow.

An interesting exercise in this context has recently been made by colleagues of
the GridKa School of computing: the material from this year (2017) has been made
publicly available on WTL\footnote{\url{https://en.wikitolearn.org/GridKa2017}}.
This constitutes an intersting example of what can be accomplished using this
platform; it is just a first example of what's possible, but an inspiring one.

Finally it should be important to evaluate, if and to which extent, the
complementary approaches to training (such as schools and dedicated web-site
courses) could be of mutual benefit (how to make them cooperate in the
development of a complete training program).

  % \item Question-and-answer web sites such as Stack Overflow 
   %(https://en.wikipedia.org/wiki/Stack_Overflow) 
   %/ Stack Exchange are very useful. 
   %(https://en.wikipedia.org/wiki/Stack_Exchange)
  % \item An expert/tutor volunteer network (office hours style) can be established.
   %\item Investing in periodic in-person, hands-on workshops can be continued.

%____________________________________________________________________
\subsection{Leveraging existing forums}

To achieve our goals for training the community, we can take advantage of
existing training forums. Resources such as conferences, workshops, and schools
(in person and online) can provide a lot of value for our training purposes with
little effort to set up. We should leverage the existing training forums that
most closely match the HEP community's needs.

Within the HEP community, there are already some working examples of dedicated
training environments that alternate between general topics and
experiment-specific topics. The LHC Physics Center (LPC) at Fermilab hosts
Hands-on Advanced Tutorial Sessions
(HATS)\footnote{\url{http://lpc.fnal.gov/programs/schools-workshops/}}
throughout the year to introduce and train participants in topics as diverse as
the latest $b$-tagging algorithms, Git/GitHub, and machine learning. These HATS
provide face-to-face time with instructors and participants at Fermilab, and
also allow remote collaborators to join in by video and complete the same online
exercises. A similar approach is in use in the CMS Data Analysis Schools
(CMSDAS)\footnote{\url{http://lpc.fnal.gov/programs/schools-workshops/cmsdas.shtml}},
a series of week-long workshops that now take place at multiple labs all over
the world and are designed to ramp up new collaborators in CMS-specific analysis
tools while providing some discussion of the physics as well.

Other examples are CERN School of
Computing\footnote{\url{https://csc.web.cern.ch/}}, CERN OpenLab Software
workshops\footnote{\url{http://openlab.cern/}} education in collaboration with
industry partners, and a series of more advanced topical training courses
provided by MPI Munich and DESY that focus on advanced programming, use of
acceleration hardware and statistical tools including machine learning. This
list includes the
Bertinoro\footnote{\url{https://web.infn.it/esc17/index.php?lang=en}}, 
GridKa\footnote{\url{http://gridka-school.scc.kit.edu/2017/}}, and
CODAS-HEP\footnote{\url{http://codas-hep.org}} Schools of Computing.

% Anecdotally, these are viewed as a great success by the CMS community in
% preparing new collaborators and might serve as a model for future efforts in
% ML training.

Over the past decade, massive open online courses (MOOCs) have been developed by
universities and private organizations. They have been well received by industry
and academia.% and are maintained by the outside community.
In addition, they provide a lot of flexibility in terms of cost and time
constraints; they are typically free and open for enrollment at any time of the
year. Since the material can be accessed at any time and revisited at any time,
they can be completed at a pace that makes sense, for example, for a physicist
who needs to learn machine leanrning in a piecemeal way.

There are a growing set of MOOCs teaching various subjects. Since there are many
options, there is a wide variety with respect to the depth of the material and
specific tools taught. Exploring these options allows us to choose which is the
right offering for the knowledge needed to work on a specific HEP experiment. We
can pick and choose modules to tailor an appropriate roadmap of skills to learn.
More difficult will be to assemble HEP specific training material not already
avaible elsewhere in an efficient and organized way, since this requires
adequate organization, volounteers and and a suitable infrastructure.

Several industry conferences already exist that bring together those in academia
and industry who are at the cutting edge of these techniques. Conferences such
as NIPS\footnote{\url{https://nips.cc}} and
PyData\footnote{\url{https://pydata.org}} provide a focused place where
attendees can learn a lot about machine learning in a short period of time.
Machine learning concepts such as current methods, tools, and problems facing
industry and academia can be learned at conferences. In addition, conferences
are an excellent networking opportunity; attendees can meet and share ideas with
fellow learners and experts. Bonds can be formed quickly at conferences that can
be maintained after the duration of the conference. These connections to the
outside community can be essential since we will be evolving training materials
to ensure that they stay relevant over time.

For example, at the time of this writing,
Coursera\footnote{\url{https://www.coursera.org/learn/machine-learning}} and
Udacity\footnote{\url{https://www.udacity.com/course/intro-to-machine-learning--ud120}}
both provide great machine learning massive open online courses at no cost.
These two courses both provide a great foundation for learning machine learning
fundamentals. However, Coursera's approach emphasizes more theory (with more
math background necessary) and uses MATLAB/Octave while Udacity's approach
emphasizes practical using machine learning techniques in a practical sense
using Python tooling.

%____________________________________________________________________
\subsection{Resources and Incentives}

Training the community is something that a large cross-section of the community
understands to be important, but finding time and effort to contribute to this
effort is not actively on the radar of most potential contributors. Providing
incentives for their participation and creating the appropriate platforms can go
a long way to reach a productive training environment. It can be as simple as
inviting someone to give a software tutorial on the subject that they are
familiar with, give a lecture or seminar or contribute to a growing knowledge
base.

Another important incentive is recognition. For younger members of the
community, having the opportunity to create a training resource, such as a
software tutorial or a knowledge base on a particular topic, is very empowering
and motivational to continue the efforts of training others. Engaging younger
members of the community is crucial to long-term success of HEP training
endeavors.

In the context of web-based training, if HSF helps to constitute a living
collection of online tutorials, we could organize regular events such as
a "Tutorial of the month", based on students votes. From a more general point
of view, since the publications are the most universal recognized items
in the research community, we should try hardly to create some sort of
training journal, with reviewed training papers.

It is also critical to incorporate training into grant proposals so that it can
be connected with other areas such as research and development. Efforts like
DIANA-HEP \footnote{\url{http://diana-hep.org/}} and aMVA4NewPhysics
\footnote{\url{https://amva4newphysics.wordpress.com/}} that combine training
and software development are good examples of such ideas in practice. More
examples of such efforts are needed.


\section{Other resources}

Software Carpentry\footnote{\url{https://software-carpentry.org}} and Data
Carpentry\footnote{\url{http://www.datacarpentry.org}} are two parts of The
Carpentries organization that collaboratively build and teach some of the basic
concepts in developing and maintaining software, and analyzing data,
respectively.  The materials that they develop and use are open, and can be
customized for science domains (e.g., HEP), or participant groups (e.g.,
undergraduates). Their model is that they offer training of trainers, and then
the trainers who have graduated can offer training under the SC/DC names, though
of course, anyone can use the SC/DC materials without doing it under the SC/DC
names.
Subjects covered in a typical Data Carpentry school are given below. This list
is taken from the curriculum of the CODATA-RDA Research Data Science Summer
School in progress in Trieste, Italy during July 2017.
(http://indico.ictp.it/event/7974/):

\begin{itemize}
\item Introduction
\item UNIX Shell programming (Software Carpentry Module)
\item GitHub (Software Carpentry Module)
\item R (Software Carpentry Module)
\item BYOD (Bring Your Own Data) best practices
\item Data Management with SQL  (Software Carpentry Module)
\item RDM Storage Management
\item Visualisation
\item Machine Learning Overview - Recommendation
\item Recommender Systems
\item Artificial Neural Networks and other Machine Learning Systems 
\item Research Computational Infrastructure 
\end{itemize}

While Software Carpentry is leveraged to build a foundation of knowledge for
later more advanced concepts, it is important to note much of this material is
developed specifically for this course and not a part of a larger Software
Carpentry program. This course focuses on a shallow but wide building of
foundational skills approach, introducing many base concepts but not going deep
into any one concept, with a through line of Open Science and Ethical Data
Usage.

\section{Conclusions}

The HEP community by and large acknowledges and recognizes the great importance
of training in the field of scientific computing. This activity should encompass
several types of {\it students}, from undergraduates, to young researchers, up
to senior physicists, all of them in need of an appropriately designed training
path in order to be proficent in their scientific endevours.

We have identified a certain number of problems that need to be overcome in
setting up an appropriate training program:
\begin{itemize}
    \item Costs and relative funding
    \item Incentives
    \item Career paths
    \item Overall organization across experiments, countries and 
    corresponding Funding Agencies
\end{itemize}

For each of these points we provide an overview of the current situation and possible proposals to build a roadmap.

%____________________________________________________________________
%____________________________________________________________________
%\begin{itemize}
%
%   \item How to motivate the community to participate in developing programs? Can some of it come from \& leadership?
%   \item How to deal with competitiveness in the field? Can’t force people to be interested in forming a community. How can we incentivize this? Building a %vibrant, welcoming community may help.
%\end{itemize}


\sloppy
\raggedright
\clearpage
\printbibliography[title={References},heading=bibintoc]

\printbibliography[title={references}]

\end{document}
