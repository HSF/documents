% This is the styling for submission to arXiv using JHEP Preprint Style

% Copyright (C) 2018, the authors, licence CC-BY-4.0.

% JHEP preprint template
\documentclass[11pt,a4paper]{article}
\usepackage{jheppub}
\notoc

\usepackage{graphicx}
\usepackage{subcaption}
%\usepackage[titletoc]{appendix}
\usepackage{xspace}

\usepackage[utf8]{inputenc}

%% Comment out before final submission
%\usepackage{lineno}  % for line numbering during review
%\linenumbers

\abstract{In 2017 the experimental High-Energy Physics community wrote a
\emph{Roadmap for HEP Software and Computing R\&D for the
2020s}\footnote{\href{https://arxiv.org/abs/1712.06982}{\emph{arXiv:1712.06982}}}.
This effort was organised by the HEP Software
Foundation\footnote{\href{https://hepsoftwarefoundation.org/}{\emph{https://hepsoftwarefoundation.org/}}}
(HSF) and was supported by more than 300 physicists from more than 100
institutes worldwide. It delivered a strategy outlining the most
important areas in which investment is needed to ensure the success of
our experimental programme. This contribution to the ESPP is an
executive summary of the most critical and relevant points raised in
that white paper.
}

\begin{document}

% HSF Document Number (only for arXiv)
\noindent
\begin{tabular*}{\linewidth}{lc@{\extracolsep{\fill}}r@{\extracolsep{0pt}}}
 %& & HSF-NOTE-2018-01 \\
 & & December 17, 2018 \\ % use \date or hardwire e.g. December 15, 2017
 & & \\
\end{tabular*}
\vspace{2.0cm}

\title{The Importance of Software and Computing to Particle Physics}

\author{A contribution from the High-Energy Physics Software Foundation
to the European Particle Physics Strategy Update 2018-2020}

\maketitle

{
\setlength{\parindent}{0cm}
Contact: Graeme A Stewart,
\href{mailto:graeme.andrew.stewart@cern.ch}{\emph{graeme.andrew.stewart@cern.ch}}
}

\newpage

\section*{The Importance of Software to the HEP Experimental
Programme}\label{the-importance-of-software-to-the-hep-experimental-programme}

The current success of the experimental High-Energy Physics programme
relies hugely on software and computing. We have developed many
innovative and unique pieces of software as a necessity, as no
off-the-shelf solutions were available. The HEP software landscape
consists of tens of millions of lines of code, from widely used community
supported packages, like ROOT and Geant4, to experiment and detector
specific codes. Thousands of physicists spend time writing this code and
the lifetimes of HEP experiments require maintaining and evolving this
software over decades. These codes are run at a massive scale. For the LHC
experiments, in particular, about 750k CPU cores are constantly employed
on the Worldwide LHC Grid running event generation, simulation,
reconstruction and analysis jobs that are critical to our physics
output. The cost of this part of our programme is very high - to
purchase just this CPU resource in a commercial cloud would cost hundreds of
millions of Euros every year. It is incumbent on us to use these
resources as efficiently as possible for physics.

\section*{Future Challenges in an Evolving Technology
Landscape}\label{future-challenges-in-an-evolving-technology-landscape}

A programme of new and upgraded accelerators and experiments will result
in far higher data rates, for both trigger and offline. We need to adapt
our data processing code and data handling infrastructure to cope with
these rates, but we do so in a landscape where the core technologies
and the scientific computing infrastructure will not be the
same as before and violate many of the assumptions made in the past.

HEP benefited for many years from the rapid evolution of microprocessor
technology and a homogeneous data centre model (x86 and Linux). While
shrinking feature sizes still lead to increases in transistor density
(Moore's Law), improvements in clock speeds have stopped and only
internal processor parallelism is improving. This fundamental shift
means that software now has to exploit more and more parallelism to
reduce processing time. Controlling memory usage and bandwidth becomes
critical to maintaining throughput. This paradigm shift from serial to
parallel execution is far complete for HEP and requires more software
developers with both specific technical expertise and physics domain
knowledge to work together.

Modern complex CPU architectures are now no longer seen as optimal for many
workloads and the performance gains on different platforms, such as GPUs or
FPGAs, have been much greater than on CPUs. Adapting our large code bases to
this new hardware landscape often requires a fundamental redesign of algorithms
from traditional serial implementations. The current set of C++ abstractions
used for two decades in both the algorithmic and framework codes needs to be
substantially rethought.

Currently HEP handles data across many geographically dispersed, but
rather homogeneous, computing centres. Our grid infrastructure is now
evolving to incorporate other resources, such as HPC centres with GPUs.
Other resources may be ephemeral and
our new data handling strategies must be able to accommodate these
resources easily. Production systems must also adapt to new workflows,
e.g. related to large scale machine learning training. Much of this new
infrastructure will be shared with other science communities as well, so
assumptions of dedicated network and storage resources only for HEP must
be relaxed.

\subsection*{Software for Future
Detectors}\label{software-for-future-detectors}

Efficient working software is required years in advance in order to
properly optimise detector design for efficient reconstruction. This is
true across all the frontiers of particle physics (energy, intensity and
cosmic). One example is future hadron colliders, which present a huge
challenge. Such accelerators would produce collisions at a pile-up of up
to 1000, requiring new detector technologies and significant advances in
reconstruction software to extract signals in a reasonable time.
Efficient reconstruction and simulation is needed and must be extremely
accurate to support a precision physics search programme. We must
anticipate analyses which deal with orders of magnitude more events,
requiring more efficient approaches to data storage and processing,
while not burdening analysts with complex interfaces. We note that such
challenges are faced now in many sectors of `big data' society and HEP
can both learn from and contribute to these domains.

\section*{Necessary Investments in
People}\label{necessary-investments-in-people}

To face the computing challenges of the future HEP programme substantial
effort is required to improve our software and computing. Ongoing
investment in R\&D, with increased funding, will be critical to extract
as much physics as possible from the data delivered by the upgraded LHC
and other detectors.

Areas for investment must focus on developing and evolving our software
so that it can adapt to the changing and uncertain hardware landscape,
that it can be sustained for decades, and that it can incorporate
advances from other fields, such as machine learning and data science.
This will require cooperation with software engineers and computer
scientists who are not traditionally from our field. The necessary
skills we need require proper, well-funded training for students and
postdocs; this training will make the contribution from HEP even more
valuable when people move outside our field. We have long lead times and
lifetimes for experiments, meaning we need long lifetimes for software
and long-term support for those who develop it. It is thus crucial that
we offer well defined career paths, including new roles such as Research
Software Engineers, for those who chose to work in these software areas,
so as to retain the necessary talent for our discipline. This is
especially true for those who chose to focus on the software engineering
of scientific code, where long-term career prospects must be viable.
This requires endorsing new initiatives to attract funding for a
sustainable and cooperative scientific software ecosystem at European
universities and labs.

\end{document}
