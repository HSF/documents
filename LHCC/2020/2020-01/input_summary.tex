

\hypertarget{summary}{%
\section{Summary}\label{summary}}

The main challenges and R\&D lines of interest have been discussed
above. There are clearly identified priority topics for the different
areas in the next few years\footnote{Longer term R\&D, into high-risk, high-reward blue sky areas,
such as quantum or neuromorphic computing, should happen at a low level,
but are almost certainly too far from production to deliver improvements
for Run 4 of the LHC.}.

\hypertarget{physics-event-generators-1}{%
\subsection{Physics Event Generators}\label{physics-event-generators-1}}

\begin{enumerate}
\def\labelenumi{\arabic{enumi}.}
\item
  \begin{quote}
  Gain a more detailed understanding of current CPU costs by accounting
  and profiling.
  \end{quote}
\item
  \begin{quote}
  Survey generator codes to understand the best way to move to GPUs and
  vectorised code, and prototype the port of the software to GPUs using
  data-parallel paradigms.
  \end{quote}
\item
  \begin{quote}
  Support efforts to optimise phase space sampling and integration
  algorithms, including the use of Machine Learning techniques such as
  neural networks.
  \end{quote}
\item
  \begin{quote}
  Promote research on how to reduce the cost associated with negative
  weight events, using new theoretical or experimental approaches.
  \end{quote}
\item
  \begin{quote}
  Promote collaboration, training, funding and career opportunities in
  the generator area.
  \end{quote}
\end{enumerate}

\hypertarget{detector-simulation-1}{%
\subsection{Detector Simulation}\label{detector-simulation-1}}

\begin{enumerate}
\def\labelenumi{\arabic{enumi}.}
\item
  \begin{quote}
  Undertake a detailed performance analysis of current production
  detector simulation code, to better understand where performance
  limitations in data and cache locality arise from.
  \end{quote}
\item
  \begin{quote}
  Improve current codes through refactoring and specialised HEP models
  to avoid bottlenecks identified.
  \end{quote}
\item
  \begin{quote}
  Develop further fast simulation models and techniques, including
  machine learning based options in addition to optimised parametric
  approaches as well as common frameworks for their tuning and
  validation.
  \end{quote}
\item
  \begin{quote}
  Undertake R\&D into GPU codes for tackling very specific, time consuming, parts of
  the simulation for HEP. Specifically calorimetry, including geometry,
  field and electro-magnetic physics processes.
  \end{quote}
\item
  \begin{quote}
  Develop integration prototypes to exercise and benchmark the
  simultaneous use of GPU specific libraries with CPU-based software
  (e.g. Geant4) to cover all particles and processes required in
  experiments simulation frameworks.
  \end{quote}
\end{enumerate}

\hypertarget{reconstruction-and-software-triggers-1}{%
\subsection{Reconstruction and Software
Triggers}\label{reconstruction-and-software-triggers-1}}

\begin{enumerate}
\def\labelenumi{\arabic{enumi}.}
\item
  \begin{quote}
  Develop fast and performant reconstruction software, making optimal
  use of hardware resources. A particularly crucial development for
  HL-LHC is that of pattern recognition solutions (including timing
  information, if available) that can withstand the increase of pile-up
  at HL-LHC. Wherever possible, solutions should be uniform for online
  and offline software, so the experiments can benefit from consistent
  event selection between trigger and final analysis and from more
  effective real-time analysis results.
  \end{quote}
\item
  \begin{quote}
  Develop solutions that can exploit accelerators (FPGA, GPU) for
  trigger and reconstruction, including general-purpose interoperability
  libraries with a focus to generalising and mitigating their
  dependencies on architect\-ure specific codes.
  \end{quote}
\item
  \begin{quote}
  Further promote activities in the area of data quality and software
  monitoring, as the quality of the recorded and reconstructed data is
  paramount for all physics analysis.
  \end{quote}
\item
  \begin{quote}
  Support efforts towards the implementation of machine learning models
  in experimental framework (C++) to enable more widespread use in
  trigger and reconstruction software.
  \end{quote}
\end{enumerate}

\hypertarget{data-analysis-1}{%
\subsection{Data Analysis}\label{data-analysis-1}}

\begin{enumerate}
\def\labelenumi{\arabic{enumi}.}
\item
  \begin{quote}
  Identify a CPU-efficient data reduction and storage model that serves
  the majority of analyses with a footprint of O(1kb/event), retaining
  the flexibility to accommodate the needs of specialised analyses.
  \end{quote}
\item
  \begin{quote}
  Streamline analysis metadata access and calibration schemes, and
  provide effective book-keeping for fractional datasets.
  \end{quote}
\item
  \begin{quote}
  Develop cross-experimental tools establishing declarative syntax
  and/or languages for analysis description, interfaced with distributed
  computing backends.
  \end{quote}
\item
  \begin{quote}
  Define and improve schemes for interoperability of end-stage
  experimental software with data science and machine learning frameworks.
  \end{quote}
\end{enumerate}

Undertaking this programme of priority topics and R\&D requires, first
and foremost, investment from funding agencies to support developments.
A new generation of developers is also needed to refresh and regenerate
the long lived software projects on which the field so heavily relies.
The transition from R\&D to production ready software is usually long
and so it is urgent to undertake this investment now, with the positive
outcomes ready to become integrated in time for HL-LHC. Regenerating the
cadre of software developers in HEP also requires support for realistic
long term career prospects for those who specialise in this vital area.
Collaboration with industry can certainly be fruitful and is to be
encouraged to allow early access to technology and communication of HEP
problems and priorities.
R\&D that is successful will become the next generation of production
software, but this will require lifetime support for maintainance and
further evolution.

Almost all the domain areas identify the use of compute accelerators,
particularly GPUs, as being a priority item, which matches our
expectation of how computing hardware will evolve. As a common problem
area, this is an obvious area in which expertise should be developed.

The stakeholders in HEP must be able to give their priorities and
feedback on R\&D areas. Each project has mechanisms for doing this,
while the HSF can help to provide overall input on prioritisation
and coordination of common cross-experiment activities.
