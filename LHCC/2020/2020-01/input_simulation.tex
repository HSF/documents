
\hypertarget{detector-simulation}{%
\section{Detector Simulation}\label{detector-simulation}}

\hypertarget{introduction-2}{%
\subsection{Introduction}\label{introduction-2}}

In HEP, data analysis, theory model evaluation and detector design
choices rely on detailed detector simulation. Since the start of the
first run the LHC experiments have produced, reconstructed, stored,
transferred and analysed hundreds of billions of simulated events. This
effort required a major fraction of WLCG's computing 
resources~\cite{ALICE-TDR-12, ATLAS-TDR-17, CMS-TDR-7, LHCb-TDR-11, WLCG-CM-Update, 
ALICE-TDR-019, LHCb-TDR-018, ATLAS-LHCC-2019-02, CMS-LHCC-2019-09}

The increase in delivered luminosity in future LHC runs, in particular
with the HL-LHC, will create unique research opportunities by collecting
an order of magnitude more data. At the same time, the demand for
detector simulation will grow accordingly so that
statistical uncertainties are kept as low as possible. However, the expected
computing resources will not suffice if current detector simulation is
used; the availability of simulated samples of sufficient size
would soon become a major limiting factor as far as new physics
discoveries, for example through precision measurements, are concerned.
Development of faster simulation, therefore, is of crucial importance
and different techniques need to be explored under the assumption that
they will provide a sufficient gain in time performance with a
negligible or acceptable loss in physics accuracy.

The Geant4 simulation toolkit has been the de facto standard for HEP
experiment simulation for physics studies over the last two decades~\cite{Geant4-2003, Geant4-2006, Geant4-2016}.
Designed in the late 1990s to cope with the simulation challenges of the
LHC era, Geant4 introduced flexible physics modelling, exploiting layers
of virtuality. This fostered progress in physics by comparison of
competing models on the same energy ranges, and allowed complementary models
to be combined to cover large energy ranges. The simulation applications
of the LHC experiments make use of this toolkit and, as such, most of
the LHC physics program relies on it. It has delivered a physics
precision that allows the experiments to perform precision analysis of
the collected data and to produce new results pushing the boundaries of
our understanding in HEP. At the same time, the code of the toolkit is
becoming old, with some parts written almost thirty years ago, making it
more and more difficult to run efficiently on modern computing
architectures. Any improvement in the critical elements of Geant4 can be
leveraged by the applications that use it. 
Improvements of physics models need to continue to take place to prevent
systematic uncertainties from simulation from becoming a dominant
factor, however, their exact influence on the precision of physics
analysis is not always easy to quantify. On the other hand, the overall
CPU performance of the current code will not improve drastically just by
relying on modern hardware or better compilers. 

Taking all those elements into account, three main development paths have emerged to be
undertaken simultaneously and these have started to be pursued. However,
effort in any one of them is far from sufficient to exploit their
individual potential. 

Firstly, it is necessary to continue to modernise
and refactor the simulation code by using more optimal, modern
programming techniques. 

%Those improvements, requiring considerable
%effort, but involving no major paradigm changes, can reasonably bring several tens
%of percent of speed-up, by improving throughput on modern CPUs,
%avoiding constant churn in data and instruction caches. This will
%certainly help to produce more simulated events, but it will not be
%sufficient to meet the high-luminosity needs of the LHC experiments. 

Secondly, different fast simulation techniques, where tracking of
individual particles in given detectors is replaced by some sort of
parameterisation and precision is traded to achieve higher event
throughput.
% (thus lowering the cost of samples overall). Techniques to
%reuse part of the simulated events multiple times have also emerged.
%New technologies (like machine learning), should certainly be
%investigated as fast simulation approaches. All of these will become
%more and more significant and will need to be seamlessly integrated with
%the detailed simulation provided by Geant4. 

Finally, the use of compute
accelerators (like GPUs or FPGAs) is the third emerging path that needs
to be undertaken and that requires intense R\&D effort.

% Discussion here of the 3 approached is quite asymmetric - maybe too many details for the first two

We will discuss
more in detail these three development axes in the following sections, 
outlining the developments identified in and since the
Community White Paper~\cite{Albrecht2019, SimFoundation2018hep}. 

\hypertarget{geant4-rd}{%
\subsection{Geant4 R\&D}\label{geant4-rd}}

The Geant4 particle transport toolkit contains advanced models for
electromagnetic and hadronic physics processes, as well as the
geometrical description of the detectors. The toolkit design principles
were to allow flexibility in future extensions of its own code, as well
as choices for particular applications. It is clear that this
flexibility comes at the price of a performance penalty. It was observed that
writing more compact and more specialised code can lead in certain
places to up to several tens of percent speed-up~\cite{GeantV-2019, GeantV-2020}. The
technical improvements to explore here consist of avoiding too many
virtual layers and improving the instruction and data locality, leading
to better cache use.

Going even further, as far as specialisation is
concerned, studies suggest that implementing compact, self-contained
transport libraries with a minimal set of physics models (for instance
only electromagnetic interaction for selected particles) with predefined
scoring (to simulate, for instance, the response of an EM calorimeter)
can also bring speed-ups to the complete application~\cite{GeantV-2019, GeantV-2020}.

As
far as architectural modifications are concerned, the design of the
Geant4 track-status machinery prevents the introduction of fine-grained
track parallelism. It is still not clear whether this kind of
parallelism can bring any substantial speed-up, due to the overheads;
however, making the Geant4 kernel and physics processes stateless would
certainly provide the necessary starting points for further R\&D in that
direction.

Most of these investigations have started and it is hoped to
see some first results on a year long time-scale. More effort, however, will
be required to integrate any eventual changes in the production code in
a timely manner.

The GeantV Vector prototype has allowed the assessment
of the achievable speedup using a novel, vectorised approach to particle
transport. The demonstrator of a full electromagnetic shower in
realistic (CMS) geometry has been implemented and the comparison to a
similar Geant4 application has been performed. The conclusion of that
work showed that the main factors in the speedup seem to include better
cache use and tighter code, while the vectorisation impact was much
smaller than hoped for. This unimpressive impact is due to a set of
factors, including the fraction of the current algorithms that could be
vectorised, unresolved challenges for the gather/scatter and tail
handling (to keep the number of events in flight within bound) needed
for large number of shapes, and of course Amdahl's Law applied to
vectorisation where some bottleneck (for example geometry navigation)
could not be tackled without a major additional long term effort~\cite{GeantV-2019, GeantV-2020}.

At the same time libraries developed in the context of the
GeantV R\&D, e.g. VecGeom, have been successfully integrated in Geant4
and ROOT benefiting the whole HEP software community~\cite{CMS01}. The
cost-over-benefit of more vectorisation of the code has not been deemed
worth the necessary investment also due to the speculation that a
complete rewrite of the physics base could become necessary to fully
exploit it. As a result investing in the optimisation and modernisation
of the Geant4 code has gained even more relevance.

\hypertarget{experiments-applications-and-optimisation-of-the-use-of-geant4}{%
\subsection{Experiment Applications and Optimised Use of
Geant4}\label{experiments-applications-and-optimisation-of-the-use-of-geant4}}

All LHC experiments have dedicated simulation frameworks making use of
the Geant4 toolkit for modelling physics processes in their detectors. Every experiment
configures and uses what is provided by Geant4 for their specific needs.
Effort has been spent by each experiment since the publication of the
Community White Paper to benchmark their 
simulation code and to maximally
exploit the options provided by the Geant4 toolkit to optimise the
performance of their applications~\cite{ATLAS-G4OPT}. All of the various handles provided
are being continuously explored and adopted when deemed useful: range
and physics cuts have been reviewed and customised by all experiments,
shower libraries adopted as baseline in the forward region by ATLAS and
CMS, stepping granularity optimised, magnetic field caching
investigated, neutron Russian roulette used for dedicated simulations.
All of these can have major impacts on simulation time and sharing
knowledge and experience between experiments can be of high benefit even
if each experiment has to find its optimal solution.

ATLAS, CMS and LHCb have integrated the Geant4 event-based
multi-threading features within their own multi-threaded frameworks,
harmonising their different architectural choices with the Geant4
implementation~\cite{CMS02, LHCb01, ATLAS01}, and have used, or are planning to
use, them in their production environment to gain access to a larger set
of distributed resources. ATLAS has been running their multi-process
simulation framework on HPCs systems for production in the
last years~\cite{ATLAS02, Benjamin:2696330} 
while CMS has pioneered the use of their multi-threaded
simulation on available resources including HPCs~\cite{CMS03}. Both
approaches have allowed the exploitation of additional computing
resources by enabling access to lower memory systems.

\hypertarget{fast-simulations}{%
\subsection{Fast Simulations}\label{fast-simulations}}

Fast simulation techniques consist of replacing the
tracking of individual particles through the detector (or part thereof),
including all the physics processes they would undergo, by a
parameterisation, where the detector response is produced directly as a
function of the incoming particle type and energy.

An example of such a
parameterisation was implemented many years ago in the context of the H1
experiment~\cite{H1Gflash}. This parameterisation, available within the
Geant4 toolkit under the name of the GFLASH library, became the starting
idea for several custom implementations, specific for other
calorimeters. Those were interfaced to experiments' simulation
applications through a hook in the Geant4 toolkit. The LHC
experiments implemented, for example, dedicated parametrised response
libraries for some of their calorimeters. Recently, an effort has been
invested in Geant4 to generalise the parameterisation formulae, and to
implement automatic procedures of tuning the parameters for specific
detector geometries. This kind of `classical' fast simulation, which
describes the shower shape using some complex mathematical
functions, with several parameters, will remain an important tool;
however, it is also clear that their overall precision for different
energy values, especially for highly granular and complex calorimeters
will always be relatively low. 

Recent developments of deep
learning-based techniques have opened up an exciting possibility of
replacing those `classical' parameterisations by trained neural networks
that would reproduce the detector response. This approach consists of
training generative models, such as Generative Adversarial Networks
(GAN), Variational Auto-Encoders (VAE) or Autoregressive Generative
Networks on the `images' of particle 
showers~\cite{ML001-GAN, aishik_ghosh_2019_3599705, GAN-ATLAS}. The energy
deposition values in the calorimeter cells are considered as `pixels' of
the images that the network is supposed to reproduce. The first studies
and prototypes have shown very promising results, however, the
generalisation of the developed models (for different calorimeters, with
different particles incident at different angles with different
energies), still requires further effort. Detailed physics validation of
those tools and understanding their capability to reproduce
fluctuations is a prerequisite towards production quality fast
simulation libraries. This work is already ongoing, but will require,
over the next few years, more effort to be invested, combining
physics and Machine Learning expertise. Given that the training of the
deep learning algorithms usually requires using sufficiently large MC
samples produced using standard techniques, the development of these novel
tools does not necessarily remove the need to speed up Geant4.

It is worth pointing out that different fast simulation techniques have
already been extensively used for LHC simulation campaigns. Effort has
been spent in the experiments frameworks to combine as much as possible
fast and detailed implementations. In particular, ATLAS and CMS are using
shower libraries in production for the modelling of
calorimeters in the forward region, in combination with the detailed
simulation of the rest of the sub-detectors. In LHCb the re-use of part
of the underlying events for specific signals of interest has been
established in production of samples when appropriate, with particular
care paid to ensure keeping under control bias on statistical
uncertainties~\cite{LHCb02}. Applicability of this technique in ATLAS and CMS could be
explored for specific cases. Similarly, the re-use of simulated or real
events to mimic the background to hard-scatter events is also being exploited,
where additional challenges exist concerning storage and I/O due to the
handling of the large minimum bias samples needed to model the
additional interactions.

All LHC experiments have explored the use of multi-objective regression
and generative adversarial networks (GANs)~\cite{GAN-ATLAS, GAN-LHCb}, they are now in the process of investigating the use of
fast simulations for other types of detectors than calorimeters, e.g.
Cerenkov based systems~\cite{Lamarr} and Time Projection Chambers~\cite{ALICE-TPC}. While implementations of fast simulations of given
sub-detectors is specific to their technology and experimental
environment and as such the responsibility of each LHC experiment, an
area of potential common investigation is frameworks for fast tuning and
validation.

The use of fully parametric response of experimental setups at the level
of reconstructed objects (tracks and clusters) for rapid evaluation of
physics reach is also exploited by the experiments with generic
frameworks for fast simulation of typical collider detectors being used~\cite{bib-Delphes, PGS4}. 
The advantage of such frameworks is that they allow
simultaneous studies for different experiments as well as their
potential use by the theory community. Experiment specific fully
parameterised simulations providing reconstructed objects compatible
with the experiment's analysis frameworks for systematic verifications
have also been emerging~\cite{Lamarr} .

It is reasonable to expect that the full variety of simulation options,
from `smeared reconstructed quantities' to parameterisation for specific
detectors to detailed Geant4, will need to be exploited by the
experiments for different tasks. Seamless ways of providing them in a
transparent and integrated way in the experiments' simulation frameworks
should continue to be pursued with Geant4 support.

\hypertarget{technical-challenges-software-and-computing-1}{%
\subsection{Technical Challenges}\label{technical-challenges-software-and-computing-1}}

The hardware landscape has always set directions as far as
software evolution is concerned. The recent adaptation of the
simulation code to multithreading (MT) turned out to be relatively
straightforward, but still took several years to adopt and validate.
Geant4 can now run in MT mode efficiently, distributing the events
between threads and sharing resources, like the geometry description
or physics data, and thus reduce the memory footprint of a simulation
while maintaining its throughput performance. The efficient utilisation
of Single Instruction Multiple Data (SIMD) architectures, on the other
hand turned out to be more challenging. Not only was a limited part of
the code vectorisable but also, as the already mentioned GeantV R\&D
has shown, overheads related to gathering the data for vector processing
presented a significant challenges (tail handling, work balancing across
threads, etc.) to efficiently (development time wise and run time wise)
profit from it~\cite{GeantV-2019, GeantV-2020}.

The most recent revolution in computing hardware is the
use of Graphics Processing Units (GPUs) for general purpose computing. Several
attempts have been made to port specific simulation code to GPUs. A few
of them turned out to be successful, leading to factors of several
hundred or a thousand speedup~\cite{GPU-MPEXS-DNA, Hybrid-Gate, Opticks}; however, they were always
limited to a very specific simulation problem, far from what is
addressed by a general HEP simulation. This application of GPUs has been
very successful in restricted domains like medical physics simulation,
neutron transport~\cite{SHIFT01} or optical photon transport~\cite{CHEP-Juno}. 
In those applications, the type of particles considered
is very limited (often just one type, with no secondary particle
production), the set of physics processes is reduced and the geometry is
often extremely simple compared to the LHC detectors. A natural
extrapolation of those approaches to a general HEP simulation is very
difficult to imagine, because the stochasticity of the simulated events
would immediately lead to divergences as far as different GPU threads
are concerned, not to mention, simply the feasibility of porting the
whole simulation to GPU code, for which specific expertise would be
needed. 

On the other hand, running only very small pieces of the
simulation application on GPUs does not seem to be efficient either, as
gathering data and transferring it from the host to the device and back again
may strongly limit any possible performance gain, similar to what was
seen with the GeantV vectorisation R\&D. The most plausible approaches
seem therefore to lead in the direction of specialised libraries (like
those described above in the context of speeding up Geant4 simulation on
the CPUs) that would perform the complete simulation of specific
sub-detectors (for instance EM calorimeter, for specific incoming
particles or Cherenkov-based detectors for the optical processes) on the
device. Such libraries could be the right compromise between the
complexity of the algorithm that needs to be ported and
the overall time that is now saved in
by the full simulation on the CPU.

An investigation of
these directions is starting now, but it will certainly require
considerable effort to be invested before being able to judge
the real impact. The first steps that are currently being undertaken
consist of implementing, on GPUs, prototypes based on VecGeom geometry
and navigation capabilities~\cite{Wenzel_2017}. If those prototypes turn out to be
successful, the next challenge will consist of adding tracking and
eventually a limited subset of physics models. While Geant4 is a toolkit
capable of addressing different modelling and simulation problems and
contains many features and capabilities allowing for user access to
detailed information for a large variety of use cases and scientific
communities, this GPU development might turn into specialised transport
modules, stripped of some features which would be expensive to implement
or support efficiently on GPUs.

Another interesting avenue consists of
exploring the application of some vendor libraries (like Nvidia Optix).
Those libraries, originally developed for ray tracing, have several
similarities with general particle transport and, if applied
successfully, could lead to major speed-ups. All these efforts certainly
require a major investment and new developers, expert in those
technologies, to join the R\&D work.

\hypertarget{other-activities}{%
\subsection{Other Activities}\label{other-activities}}

Common digitisation efforts would be desirable among experiments, with
advanced high-performance generic examples, which experiments could use
as a basis to develop their own code. Nevertheless the large variety of
detector technologies used reduces its
applicability. Digitisation is not yet a limiting factor, in terms of
CPU requirements, so developments in this area have not been a priority.

Simulated samples are often processed through the reconstruction as real
data. As such in some cases they require the emulation of hardware
triggers. Hardware triggers are based on very specific custom devices
and a general approach does not seem very feasible, even if some
parametrisation could be generalised.

\hypertarget{outlook}{%
\subsection{Outlook}\label{outlook}}

It is important to realise that there is a considerable
risk that simulation becomes a major limiting factor as far as new
physics discoveries are concerned if no serious effort is invested in
R\&D. Concerted effort of different experts in physics,
Machine Learning and GPUs is needed. There are too many unknowns to focus on only
one direction, so a lot of different prototyping is needed. Development of Geant4,
while working on fast simulation algorithms and
doing extensive R\&D on leveraging compute accelerators is required. While the R\&D
activities are taking place, one can not forget that it is still
necessary to support the existing code and make sure that sufficient
funding and staffing is provided for maintenance and development of
physics algorithms, as well as for adapting the code to any updated CPU
hardware, operating systems and new compilers.

It is also important to stress once again the need for a continuous
integration activity of new runtime performance
improvements and solutions into the Geant4 code base and experiments
applications to profit from any improvement as quickly as possible. An
effort needs to be invested into making it possible to evolve it to meet
the HL-LHC needs without compromising the production quality of the
Geant4 code used by the experiments. We need to ensure that new
developments resulting from the R\&D programs can be tested with
realistic prototypes and, if successful, then integrated, validated, and
deployed in a timely fashion in Geant4 and adopted by the experiments.
The strategy adopted in the successful integration in Geant4 of VecGeom
libraries developed in the context of the GeantV R\&D can provide a
working example of how to proceed to provide some incremental
improvements to existing applications.

No single solution appears at the moment to provide by itself the
processing gains required by HL-LHC, nevertheless if a novel approach
emerges that could do so it should be pursued and carefully evaluated in
terms of gains against the disruption of a complete re-implementation of
how events are simulated.
